%%%%%%%%%%%%%%%%%%%%%%%%%%%%%%%%%%%%%%%%%%%%%%%%%%%%%%%%%%%%%%%%%%%%%%%%%%%%%%%
\section{Software Design}
\label{sec:software}
%%%%%%%%%%%%%%%%%%%%%%%%%%%%%%%%%%%%%%%%%%%%%%%%%%%%%%%%%%%%%%%%%%%%%%%%%%%%%%%

The OpenMC Python API's \texttt{openmc.mgxs} module was designed to generate multi-group cross sections. The \texttt{openmc.mgxs} module is built atop the underlying core features in the rest of the API to support a seamless interface for both input generation and downstream data processing of MGXS from Python. In particular, one may specify the MGXS to compute and the \texttt{openmc.mgxs} module will construct the necessary \texttt{Tally} objects. The \texttt{Tally} objects may be easily exported to XML input files for OpenMC, and used to containerize and process the tally data produced by an OpenMC simulation. The \texttt{openmc.mgxs} module thereby leverages the software stack (\textit{e.g.}, tally arithmetic, Pandas DataFrames) provided by the Python API.

The \texttt{openmc.mgxs} module uses an object-oriented design based on an abstract \texttt{MGXS} class with subclasses for different reaction types. The \texttt{MGXS} subclasses are itemized in \autoref{tab:mgxs-types} and compute macroscopic or microscopic multi-group constants in one or more arbitrary energy group structures from MC tallies. The \texttt{openmc.mgxs} module also includes a \texttt{Library} class which automates the construction of \texttt{MGXS} objects for different group structures, spatial domains, and reaction types.

\begin{table}[h!]
  \centering
  \caption{The multi-group cross section types implemented by the \texttt{openmc.mgxs} module.}
  \small
  \label{tab:mgxs-types} 
  \vspace{6pt}
  \begin{tabular}{l l}
  \toprule
  \textbf{Class} &
  \textbf{Description} \\
  \midrule
  \texttt{AbsorptionXS} & Absorption \\
  \texttt{CaptureXS} & Radiative capture \\
  \texttt{Chi} & Fission emission spectrum \\
  \texttt{FissionXS} & Fission \\
  \texttt{InverseVelocity} & Inverse neutron velocity \\
  \texttt{KappaFissionXS} & Fission energy release \\
  \texttt{MultiplicityMatrixXS} & Scattering multiplicity matrix \\
  \texttt{NuFissionMatrixXS} & Fission production matrix \\
  \texttt{ScatterXS} & Scattering \\
  \texttt{ScatterMatrixXS} & Scattering matrix \\
  \texttt{ScatterProbabilityMatrixXS} & Scattering probability matrix \\
  \texttt{TotalXS} & Total collision \\
  \texttt{TransportXS} & Transport-corrected total collision \\
  \bottomrule
\end{tabular}
\end{table}


%%%%%%%%%%%%%%%%%%%%%%%%%%%%%
\subsection{XML Input Generation}
\label{subsec:xml-inputs}

%\begin{itemize}[noitemsep]
%\item workflow: create Python model, create \texttt{Library}, export to XML
%\item code snippet?
%\end{itemize}

%%%%%%%%%%%%%%%%%%%%%%%%%%%%%
\subsection{Data Processing}
\label{subsec:xml-inputs}

\begin{itemize}[noitemsep]
\item refer to PyAPI features that are ``inherited'' by \texttt{openmc.mgxs}:
  \begin{itemize}[noitemsep]
  \item Pandas DataFrames, tally arithmetic/slicing/merging, ...
  \end{itemize}
\end{itemize}

%%%%%%%%%%%%%%%%%%%%%%%%%
\subsection{Data Storage}
\label{subsec:xml-inputs}

The \texttt{openmc.mgxs} module was developed with general design principles to generate MGXS for any multi-group neutron transport code. Although the module does not explicitly support any multi-group codes, it can export MGXS data to a variety of data storage formats, including Comma-Separated Values (CSV) and HDF5. The exported MGXS files may be easily transformed into the database or input files required by a particular multi-group code.