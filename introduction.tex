%%%%%%%%%%%%%%%%%%%%%%%%%%%%%%%%%%%%%%%%%%%%%%%%%%%%%%%%%%%%%%%%%%%%%%%%%%%%%%%
\section{Introduction}
\label{sec:intro}
%%%%%%%%%%%%%%%%%%%%%%%%%%%%%%%%%%%%%%%%%%%%%%%%%%%%%%%%%%%%%%%%%%%%%%%%%%%%%%%

The past two decades have seen growing interest in Monte Carlo (MC) as a means to generate multigroup cross section (MGXS) libraries. Most MC-based MGXS generation schemes to date focus on generating few-group constants for coarse-mesh diffusion codes. These schemes aim to improve the accuracy of standard diffusion codes for analysis of atypical core configurations for which the simplifications made by multilevel deterministic MGXS generation methods are not necessarily applicable. These efforts replace the separate resonance self-shielding and deterministic lattice physics calculation steps in multistep approaches with fully detailed MC calculations of each assembly to compute the few-group constants needed by whole-core diffusion codes. The widely used Serpent code\cite{leppanen2015serpent} has led this trend over the past decade, and a few authors have applied the MCNP\cite{pounders2006stochastically} and McCARD\cite{shim2008generation} codes in a similar fashion. This paper presents new capabilities introduced in the OpenMC\cite{romano2015openmc} particle transport code for multigroup cross-section generation for fine-mesh multigroup neutron transport applications.

This paper is organized as follows. \cref{sec:mg-theory} summarizes the key multigroup constants required by deterministic multigroup codes. \cref{sec:mgxs-mc} details the mathematical formulation for stochastic integration of each of the multigroup constants currently computable by OpenMC. \cref{sec:openmc} highlights the core OpenMC features that provide the foundation for the \texttt{openmc.mgxs} module for MGXS generation introduced in \cref{sec:design}. \cref{sec:validate} uses the OpenMOC\cite{boyd2014openmoc} multigroup transport code to validate the MGXS generated by OpenMC. \cref{sec:conclusions} summarizes the conclusions.
