%%%%%%%%%%%%%%%%%%%%%%%%%%%%%%%%%%%%%%%%%%%%%%%%%%%%%%%%%%%%%%%%%%%%%%%%%%%%%%%
\section{Introduction}
\label{sec:intro}
%%%%%%%%%%%%%%%%%%%%%%%%%%%%%%%%%%%%%%%%%%%%%%%%%%%%%%%%%%%%%%%%%%%%%%%%%%%%%%%

The last two decades have seen growing interest in Monte Carlo as a means to generate multi-group cross section (MGXS) libraries. Most MC-based MGXS generation schemes to date focus on generating few-group constants for coarse mesh diffusion codes. These schemes aim to improve the accuracy of standard diffusion codes for analysis of atypical core configurations for which the simplifications made by multi-level deterministic MGXS generation methods are not necessarily applicable. These efforts replace the separate resonance self-shielding and deterministic lattice physics calculation steps in multi-step approaches with fully-detailed MC calculations of each assembly to compute the few-group constants needed by whole core diffusion codes. The widely used Serpent code\cite{leppanen2007serpent} has led this trend over the last decade, and a few authors have applied the MCNP\cite{pounders2006stochastically} and McCARD\cite{shim2008generation} codes in a similar fashion.

OpenMC\cite{romano2015openmc}.

-section on how to compute chi
-section on scattering matrices
  -simple formulation
  -consistent formulation
