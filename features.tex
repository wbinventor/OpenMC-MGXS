%%%%%%%%%%%%%%%%%%%%%%%%%%%%%%%%%%%%%%%%%%%%%%%%%%%%%%%%%%%%%%%%%%%%%%%%%%%%%%%
\section{Features}
\label{sec:features}
%%%%%%%%%%%%%%%%%%%%%%%%%%%%%%%%%%%%%%%%%%%%%%%%%%%%%%%%%%%%%%%%%%%%%%%%%%%%%%%


%%%%%%%%%%%%%%%%%%%%%%%%%%%%%%%%%%%%%%%%%%%%
\subsection{Energy Condensation}
\label{subsec:energy-condense}

The module supports energy condensation in downstream data processing which is useful for exploring approximation bias in various energy group structures. For example, MGXS may be computed in some ``fine'' energy group structure and the tally data subsequently condensed to some coarser group structure \texttt{coarse_groups} for multi-group calculations with the \texttt{MGXS.get_condensed_xs(coarse_groups)} Python class method. Energy condensation may be performed to arbitrarily defined coarse group structures provided the coarse group boundaries coincide with boundaries in the fine group structure.


%%%%%%%%%%%%%%%%%%%%%%%%%%%%%%%%%%%%%%%%%%%%
\subsection{Pin-Wise Spatial Homogenization}
\label{subsec:pinwise-homogenize}

The \texttt{openmc.mgxs} module is designed to perform spatial homogenization on heterogeneous tally meshes for fine-mesh transport codes. In OpenMC parlance, MGXS may be computed for material, cell or universe spatial domains. In addition, the module supports MGXS calculations for repeated cell instances using distribcell spatial tally domains\cite{lax2014distribcell}. Spatial homogenization across some subset of distributed cell instances \texttt{cell_instances} can be performed using the \texttt{MGXS.get_subdomain_avg_xs(cell_instances)} Python class method.

%The \texttt{openmc.mgxs} module may also perform spatial homogenization on structured Cartesian tally meshes for coarse mesh multi-group calculations..


%%%%%%%%%%%%%%%%%%%%%%%%%%%%%%%%%%%%%%%%%%%%
\subsection{Microscopic MGXS}
\label{subsec:micro-macro}

I hear it is unique that we can calculate micros?


%%%%%%%%%%%%%%%%%%%%%%%%%%%%%%%
\subsection{Scattering Moments}
\label{subsec:scatt-moments}

Up to tenth order scattering moment matrices.


%%%%%%%%%%%%%%%%%%%%%%%%%%%%%%
\subsection{Delayed Constants}
\label{subsec:delay-constants}

\begin{table}[h!]
  \centering
  \caption{The delayed multi-group cross section types implemented by the \texttt{openmc.mgdxs} module.}
  \small
  \label{tab:mdgxs-types} 
  \vspace{6pt}
  \begin{tabular}{l l}
  \toprule
  \textbf{Class} &
  \textbf{Description} \\
  \midrule
  \texttt{Beta} & Delayed neutron fraction \\
  \texttt{ChiDelayed} & Delayed fission spectrum \\
  \texttt{DelayedNuFissionXS} & Fission delayed neutron production \\
  \texttt{DecayRate} & Delayed neutron precursors decay rate \\
  \bottomrule
\end{tabular}
\end{table}

%%%%%%%%%%%%%%%%%%%%%%%%%%%%%%
\subsection{Delayed Constants}
\label{subsec:delay-constants}


%%%%%%%%%%%%%%%%%%%%%%%%%%%%%%%%%%%%%%%%
\subsection{Isotropic in Lab Scattering}
\label{subsec:iso-in-lab}

A unique option for isotropic in lab scattering is implemented in the OpenMC code. The isotropic in lab -- abbreviated as \texttt{iso-in-lab} -- feature may be useful to quantify the ability of multi-group codes to capture anisotropic scattering effects with higher order scattering matrices or transport correction schemes. The iso-in-lab scattering feature is implemented as an optional attribute for each nuclide or element in a simulation. When iso-in-lab scattering is specified for a nuclide or element, the outgoing neutron energy is sampled from the scattering laws prescribed by the continuous energy cross section library, but the outgoing neutron direction of motion is sampled from an isotropic in lab distribution. 