%%%%%%%%%%%%%%%%%%%%%%%%%%%%%%%%%%%%%%%%%%%%%%%%%%%%%%%%%%%%%%%%%%%%%%%%%%%%%%%
\section{MGXS Generation with Monte Carlo}
\label{sec:mgxs-mc}
%%%%%%%%%%%%%%%%%%%%%%%%%%%%%%%%%%%%%%%%%%%%%%%%%%%%%%%%%%%%%%%%%%%%%%%%%%%%%%%


%%%%%%%%%%%%%%%%%%%%%%%%%%%%%%%%%%%%%%%%%%%%%%%%%%%
\subsection{Tally Types Needed for MGXS Generation}
\label{subsec:tally-types}

This section describes how multi-group cross sections may be computed using stochastic integration. In particular, this section outlines the types of OpenMC tallies needed to generate MGXS -- including the scores, filters and estimators for each tally -- and the arithmetic combinations used to combine different tallies.

%%%%%%%%%%%%%%%%%%%%%%%%%%%%%%%%%%%%%%
\subsubsection{Inner Product Notation}
\label{subsubsec:tally-types-notation}

The following sections use angle bracket notation $\langle \cdot , \cdot \rangle$ to represent inner products in phase space. This may correspond to integrals over incoming and/or outgoing energy, space, and angle. Using this notation, a tally estimator for reaction rate $x$ is represented as follows: 

\begin{equation}
\label{eqn:inner-prod}
\langle \Sigma_x, \psi \rangle = \int_{V} \int_{S} \int_{E} \Sigma_{x}(\mathbf{r},E)\psi(\mathbf{r},E,\mathbf{\Omega}) \mathrm{d}E\mathrm{d}\mathbf{\Omega}\mathrm{d}\mathbf{r}
\end{equation}

\noindent This notation is specialized throughout this section with subscripts to indicate the subsets of phase space that are integrated over in the inner product. In particular, subscript $k$ refers to a volume integral over $V_{k}$ for some region of space $k$ for spatial homogenization, while subscript $g$ corresponds to an integral over energies with $E \in [E_{g}, E_{g-1}]$ for energy condensation. For example, the microscopic reaction rate for reaction $x$ by nuclide $i$ is denoted as:

\begin{equation}
\label{eqn:angle-rxn-rate}
\langle \sigma_{x,i}, \psi \rangle_{k,g} = \int_{\mathbf{r} \in V_{k}} \int_{4\pi} \int_{E_{g}}^{E_{g-1}} \sigma_{x,i}(\mathbf{r},E)\psi(\mathbf{r},E,\mathbf{\Omega}) \mathrm{d}E\mathrm{d}\mathbf{\Omega}\mathrm{d}\mathbf{r}
\end{equation}

\noindent The inner product of a function with unity, such as the spatially-homogenized and energy-integrated flux is denoted by:

\begin{equation}
\label{eqn:angle-flux}
\langle \psi \rangle_{k,g} \equiv \langle \psi, \mathbb{1} \rangle_{k,g} = \int_{\mathbf{r} \in V_{k}} \int_{4\pi} \int_{E_{g}}^{E_{g-1}} \psi(\mathbf{r},E,\mathbf{\Omega}) \mathrm{d}E\mathrm{d}\mathbf{\Omega}\mathrm{d}\mathbf{r}
\end{equation}

%Finally, the superscripts $a$ and $t\ell$ are given to those inner products computed with analog and track-length estimators, respectively -- \textit{i.e.}, $\langle \cdot,\cdot \rangle^{a}$ is an analog tally estimator and $\langle \cdot,\cdot \rangle^{t\ell}$ is a track-length tally estimator of the corresponding inner products.

%%%%%%%%%%%%%%%%%%%%%%%%%%%%%%%%%%%%%%%%%%%%%%
\subsubsection{General Reaction Cross Section}
\label{subsubsec:tally-types-gen-xs}

A general spatially-homogenized and energy condensed macroscopic multi-group cross section for reaction $x$, spatial zone $k$ and energy group $g$ is simply the ratio of the group-wise reaction rates $\langle \Sigma_{x}, \psi \rangle_{k,g}$ and fluxes $\langle \psi \rangle_{k,g}$:

\begin{equation}
\label{eqn:general-macro}
\hat{\Sigma}_{x,k,g} = \frac{\langle \Sigma_{x}, \psi \rangle_{k,g}}{\langle \psi \rangle_{k,g}}
\end{equation}

\noindent Likewise, a microscopic MGXS for nuclide $i$ can be computed as follows:

\begin{equation}
\label{eqn:general-micro}
\hat{\sigma}_{x,i,k,g} = \frac{\langle \sigma_{x,i}, \psi \rangle_{k,g}}{\langle \psi \rangle_{k,g}}
\end{equation}

These estimators are used for reaction types which are only dependent on the incoming energy of a neutron, such as total and radiative capture reactions.

%%%%%%%%%%%%%%%%%%%%%%%%%%%%%%%%%%%
\subsubsection{Total Cross Section}
\label{subsubsec:tally-types-tot-xs}

The total macroscopic cross section $\Sigma_{t}$ is a special case of \autoref{eqn:general-micro}, with the total collision rate substituted into the numerator:

\begin{equation}
\label{eqn:total-macro}
\hat{\Sigma}_{t,k,g} = \frac{\langle \Sigma_{t}, \psi \rangle_{k,g}}{\langle \psi \rangle_{k,g}}
\end{equation}

A transport correction is often used to incorporate anisotropic scattering effects into the transport equation with an isotropic scattering kernel. An expression for the in-scatter approximation \cite{yamamoto2008simplified} to the transport correction is computed with an OpenMC tally for the first Legendre scattering moment:

\begin{equation}
\label{eqn:sigs1}
\langle \Sigma_{s1}, \psi \rangle_{k,g'\rightarrow g} = \int_{\mathbf{r} \in V_{k}} \int_{4\pi} \int_{E_{g}}^{E_{g-1}} \int_{E_{g'}}^{E_{g'-1}} \Sigma_{s1}(\mathbf{r},E'\rightarrow E)\psi(\mathbf{r},E',\mathbf{\Omega}) \mathrm{d}E'\mathrm{d}E\mathrm{d}\mathbf{\Omega}\mathrm{d}\mathbf{r}
\end{equation}

\noindent An analog estimator must be used in OpenMC since the tally includes an integral over the outgoing neutron energy. The spatially-homogenized and energy condensed transport-corrected total cross section is computed by summing over all incoming energy groups:

\begin{equation}
\label{eqn:transport-corr-macro}
\hat{\Sigma}_{tr,k,g} = \displaystyle\sum\limits_{g'=1}^{G} \langle{\Sigma_{s1}, \psi \rangle_{k,g'\rightarrow g}}
\end{equation}

\noindent The transport correction is then subtracted from the group-wise total collision rate and normalized by the flux to compute the transport-corrected total cross section:

\begin{equation}
\label{eqn:sigt-transport-macro}
\hat{\tilde{\Sigma}}_{t,k,g} = \frac{\langle \Sigma_{t}, \psi \rangle_{k,g} - \hat{\Sigma}_{tr,k,g}}{\langle \psi \rangle_{k,g}}
\end{equation}

Since the transport correction must be computed using an analog estimator, the total collision and flux in \autoref{eqn:sigt-transport-macro} must also be computed with analog estimators.

%%%%%%%%%%%%%%%%%%%%%%%%%%%%%%%%%%%
\subsubsection{Multiplicity Matrix}
\label{subsubsec:tally-types-mult-mat}

The multiplicity is calculated as

\begin{equation}
\langle \upsilon \sigma_{s,g'\rightarrow g} \phi \rangle = \int_{r \in D} dr \int_{4\pi} d\Omega' \int_{E_{g'}}^{E_{g'-1}} dE' \int_{4\pi} d\Omega \int_{E_g}^{E_{g-1}} dE \; \sum_i \upsilon_i \sigma_i (r, E' \rightarrow E, \Omega' \cdot \Omega) \psi(r, E', \Omega') 
\end{equation}

\begin{equation}
\langle \upsilon \sigma_{s,g'\rightarrow g} \phi \rangle = \int_{r \in D} dr \int_{4\pi} d\Omega' \int_{E_{g'}}^{E_{g'-1}} dE' \int_{4\pi} d\Omega \int_{E_g}^{E_{g-1}} dE \; \sum_i \upsilon_i \sigma_i (r, E' \rightarrow E, \Omega' \cdot \Omega) \psi(r, E', \Omega')
\end{equation}

\begin{equation}
\langle \sigma_{s,g'\rightarrow g} \phi \rangle = \int_{r \in D} dr \int_{4\pi} d\Omega' \int_{E_{g'}}^{E_{g'-1}} dE' \int_{4\pi} d\Omega \int_{E_g}^{E_{g-1}} dE \; \sum_i \upsilon_i \sigma_i (r, E' \rightarrow E, \Omega' \cdot \Omega) \psi(r, E', \Omega') \\
\end{equation}

\begin{equation}
\upsilon_{g'\rightarrow g} = \frac{\langle \upsilon \sigma_{s,g'\rightarrow g} \rangle}{\langle \sigma_{s,g'\rightarrow g} \rangle}
\end{equation}

\noindent where $\upsilon_i$ is the multiplicity for the $i^{th}$ reaction.


%%%%%%%%%%%%%%%%%%%%%%%%%%%%%%%%%%%%%%%%%%%%%
\subsubsection{Scattering Probability Matrix}
\label{subsubsec:tally-types-scatt-prob-mat}


%%%%%%%%%%%%%%%%%%%%%%%%%%%%%%%%%
\subsubsection{Scattering Matrix}
\label{subsubsec:tally-types-scatt-mat}

The isotropic scattering matrix is computed with an inner product of scattering reactions over both incoming and outgoing energies. An analog estimator must be used since the integral is dependent on the neutron's outgoing energy. Similar to the first Legendre moment in \autoref{eqn:sigs1}, the isotropic scattering moment is given by the following expression:

\begin{equation}
\label{eqn:sigs0}
\langle \Sigma_{s0}, \psi \rangle_{k,g'\rightarrow g} = \int_{\mathbf{r} \in V_{k}} \int_{4\pi} \int_{E_{g}}^{E_{g-1}} \int_{E_{g'}}^{E_{g'-1}} \Sigma_{s0}(\mathbf{r},E'\rightarrow E)\psi(\mathbf{r},E,\mathbf{\Omega}) \mathrm{d}E'\mathrm{d}E\mathrm{d}\mathbf{\Omega}\mathrm{d}\mathbf{r}
\end{equation}

\noindent The isotropic scattering matrix is then:

\begin{equation}
\label{eqn:scatt-macro}
\hat{\Sigma}_{s,k,g'\rightarrow g} = \frac{\langle \Sigma_{s0}, \psi \rangle_{k,g'\rightarrow g}}{\langle \psi \rangle_{k,g'}}
\end{equation}

\noindent The transport correction in \autoref{eqn:transport-corr-macro} can be applied by subtracting it from the diagonal elements in the matrix to compute the transport-corrected scattering matrix:

\begin{equation}
\label{eqn:scatt-trans-macro}
\hat{\tilde{\Sigma}}_{s,k,g'\rightarrow g} = \frac{\langle \Sigma_{s0}, \psi \rangle_{k,g'\rightarrow g} - \delta_{g,g'} \hat{\Sigma}_{tr,k,g}}{\langle \psi \rangle_{k,g'}}
\end{equation}

%To incorporate the effect of neutron multiplication from $(n,xn)$ reactions in the above relation, the nu parameter can be set to `True`.

An alternative ``consistent'' form of the scattering matrix can be computed as the product of the scatter cross section and group-to-group scattering probabilities. Unlike the formulation in \autoref{eqn:scatt-macro}, the consistent formulation is computed from the groupwise scattering cross section which uses a track-length estimator. This ensures that reaction rate balance is exactly preserved with a total cross section (\autoref{eqn:total-macro} computed using a tracklength estimator.

For a scattering probability matrix $P_{s,\ell,g'\rightarrow g}$ and scattering cross section $\sigma_s (r, E)$ for incoming energy group $[E_{g'},E_{g'-1}]$ and outgoing energy group $[E_g,E_{g-1}]$, the Legendre scattering moments are calculated as:

\begin{equation}
\label{eqn:scatt-mat-consistent}
\Sigma_{s,\ell,g'\rightarrow g} = \sigma_s (r, E) \times P_{s,\ell,g'\rightarrow g}
\end{equation}

To incorporate the effect of neutron multiplication from $(n,xn)$ reactions ...

\begin{equation}
\label{eqn:nuscatt-mat-consistent}
\sigma_{s,\ell,g'\rightarrow g} = \upsilon_{g'\rightarrow g} \times \sigma_s (r, E) \times P_{s,\ell,g'\rightarrow g}
\end{equation}

Add discussion about consistent scattering formulation ... and about Legendre moments ...

%%%%%%%%%%%%%%%%%%%%%%%%%%%%%%%%%%%%%%%%%%%%%%%%
\subsubsection{Fission Production Cross Section}
\label{subsubsec:tally-types-fiss-prod}

The fission product macroscopic cross section $\nu\Sigma_{f}$ may be treated as a special case of \autoref{eqn:general-micro}, with estimators for the fission production rate and flux:

\begin{equation}
\label{eqn:nu-fiss-macro}
\nu\hat{\Sigma}_{f,k,g} = \frac{\langle \nu\Sigma_{f}, \psi \rangle_{k,g}}{\langle \psi \rangle_{k,g}}
\end{equation}

%%%%%%%%%%%%%%%%%%%%%%%%%%%%%%%%%%%%%%%
\subsubsection{Fission Energy Spectrum}
\label{subsubsec:tally-types-chi}

Unlike the fission production cross section, the fission spectrum is dependent on the outgoing neutron energy and must be computed with analog estimators. The fission production matrix from group $g'$ into group $g$ is given by the following inner product:

\begin{equation}
\label{eqn:nu-fiss-energies}
\langle \nu\Sigma_{f}, \psi \rangle_{k,g'\rightarrow g} = \int_{\mathbf{r} \in V_{k}} \int_{4\pi} \int_{E_{g}}^{E_{g-1}} \int_{E_{g'}}^{E_{g'-1}} \nu\Sigma_{f}(\mathbf{r},E'\rightarrow E)\psi(\mathbf{r},E,\mathbf{\Omega}) \mathrm{d}E'\mathrm{d}E\mathrm{d}\mathbf{\Omega}\mathrm{d}\mathbf{r}
\end{equation}

\noindent The fission spectrum can then be computed from this tally by summing over incoming and outgoing energy groups:

\begin{equation}
\label{eqn:chi}
\hat{\chi}_{k,g} = \frac{\displaystyle\sum\limits_{g'=1}^{G} \langle \nu\Sigma_{f}, \psi \rangle_{k,g'\rightarrow g}}{\displaystyle\sum\limits_{g=1}^{G} \displaystyle\sum\limits_{g'=1}^{G} \langle \nu\Sigma_{f}, \psi \rangle_{k,g'\rightarrow g}}
\end{equation}

\noindent This expression for the fission spectrum will result in a normalized discrete probability distribution for the energy of neutrons emitted from fission.

%%%%%%%%%%%%%%%%%%%%%%%%%%%%%%%%%%%%%%%%
\subsubsection{Delayed Neutron Fraction}
\label{subsubsec:tally-types-beta}

%%%%%%%%%%%%%%%%%%%%%%%%%%%%%%%%%%%%%%%%%%%%%%%%%%%%
\subsubsection{Delayed Neutron Precursor Decay Rate}
\label{subsubsec:tally-types-lambda}

%%%%%%%%%%%%%%%%%%%%%%%%%%%%%%%%%%%%%%%%%%%%%%%%%%%%%%%%
\subsubsection{Delayed Fission Production Cross Section}
\label{subsubsec:tally-types-delay-fiss-prod}

%%%%%%%%%%%%%%%%%%%%%%%%%%%%%%%%
\subsubsection{Inverse Velocity}
\label{subsubsec:tally-types-inv-vel}


%%%%%%%%%%%%%%%%%%%%%%%%%%%%%%%%%%%%%%%%%%%%%%%%%%%
\subsection{Summary}
\label{subsec:tally-types-summary}

The tallies needed to generate MGXS libraries were outlined in detail in the preceding sections, and are summarized in \autoref{tab:tally-types}. The scores and filters correspond to the notation used by the OpenMC code to describe the scoring function and integration bounds. The energy group structure for energy condensation is specified by \texttt{energy} and/or \texttt{energyout} filters in the table. A material, cell, universe or mesh domain is specified by an appropriate filter for spatial homogenization.

\begin{table}[h!]
  \centering
  \caption[Tally types for MGXS generation]{The types of tallies used in MGXS generation with OpenMC.}
  \scriptsize
  \label{tab:tally-types}
  \vspace{6pt}
  \begin{tabular}{ m{2.3cm} m{1.2cm} m{2cm} m{2.5cm} l }
  \toprule
  {\bf Name} &
  {\bf Symbol} &
  {\bf Tally} &
  {\bf Score} &
  {\bf Filters} \\

  \midrule

  \multirow{2}{*}{\bf General} & \multirow{2}{*}{$\Sigma_{x,k,g}$} & $\langle \Sigma_{x}, \psi \rangle_{k,g}$ & reaction $x$ & spatial domain\footnotemark, \texttt{energy} \\
  \cline{3-5}
  & & $\langle \psi \rangle_{k,g}$ & {\texttt{flux}} & spatial domain\ref{foot:domain}, \texttt{energy} \\

  \midrule

  \multirow{2}{*}{\bf Total} & \multirow{2}{*}{$\Sigma_{t,k,g}$} & $\langle \Sigma_{t}, \psi \rangle_{k,g}$ & \texttt{total} & spatial domain, \texttt{energy} \\
  \cline{3-5}
  & & $\langle \psi \rangle_{k,g}$ & \texttt{flux} & spatial domain\ref{domain}, \texttt{energy} \\

  \midrule

  \multirow{3}{*}{\parbox{1.5cm}{\bf Radiative Capture}} & \multirow{3}{*}{$\Sigma_{\gamma,k,g}$} & $\langle \Sigma_{a}, \psi \rangle_{k,g}$ & \texttt{absorption} & spatial domain\ref{foot:domain}, \texttt{energy} \\
  \cline{3-5}
  & & $\langle \Sigma_{f}, \psi \rangle_{k,g}$ & \texttt{fission} & spatial domain\ref{foot:domain}, \texttt{energy} \\
  \cline{3-5}
  & & $\langle \psi \rangle_{k,g}$ & \texttt{flux} & spatial domain\ref{foot:domain}, \texttt{energy} \\

  \midrule

  \textbf{\parbox{1.5cm}{\bf Transport Correction}} & $\Sigma_{tr,k,g}$ & $\langle \Sigma_{s1}, \psi \rangle_{k,g'\rightarrow g}$ & \texttt{(nu-)scatter-0} & spatial domain\ref{foot:domain}, \texttt{energyout} \\

  \midrule

  \multirow{2}{*}{\parbox{1.5cm}{\bf Scattering Matrix}} & \multirow{2}{*}{$\Sigma_{s,k,g'\rightarrow g}$} & $\langle \Sigma_{s0}, \psi \rangle_{k,g'\rightarrow g}$ & \texttt{(nu-)scatter-0} & spatial domain\ref{foot:domain}, \texttt{energy}, \texttt{energyout} \\
  \cline{3-5}
  & & $\langle \psi \rangle_{k,g}$ & \texttt{flux} & spatial domain\ref{foot:domain}, \texttt{energy} \\

  \midrule

  \multirow{2}{*}{\parbox{1.5cm}{\bf Fission \hspace{1cm} Production}} & \multirow{2}{*}{$\nu\Sigma_{f,k,g}$} & $\langle \nu\Sigma_{f}, \psi \rangle_{k,g}$ & \texttt{nu-fission} & spatial domain\ref{foot:domain}, \texttt{energy} \\
  \cline{3-5}
  & & $\langle \psi \rangle_{k,g}$ & \texttt{flux} & spatial domain\ref{foot:domain}, \texttt{energy} \\

  \midrule
  
  \parbox{1.5cm}{\parbox{1.2cm}{\bf Fission Spectrum}} & $\chi_{k,g}$ & $\langle \nu\Sigma_{f}, \psi \rangle_{k,g'\rightarrow g}$ & \texttt{nu-fission} & spatial domain\ref{foot:domain}, \texttt{energy}, \texttt{energyout} \\

  \midrule

  \parbox{1.5cm}{\parbox{1.9cm}{\bf Delayed Fission Spectrum}} & & & \\

  \midrule

  \parbox{1.5cm}{\parbox{2.3cm}{\bf Delayed Prescursor Decay Rate}} & & & \\

  \midrule

  \parbox{1.5cm}{\parbox{2cm}{\bf Delayed Neutron Fraction}} & & & \\

  \midrule

  \parbox{1.5cm}{\parbox{1.2cm}{\bf Inverse Velocity}} & & & \\

  \midrule

\end{tabular}
\end{table}

\footnotetext{\label{foot:domain} A \texttt{material}, \texttt{cell}, \texttt{distribcell}, \texttt{universe} or \texttt{mesh} filter.}


%%%%%%%%%%%%%%%%%%%%%%%%%%%%%
\subsection{Tally Estimators}
\label{subsec:tally-est}

\begin{itemize}[noitemsep]
\item Table of the tally estimators acceptable for each MGXS type
\item Mention tally estimators can be toggled? Or discuss in software design?
\item Add footnote mentioning consistent scattering formulation
\end{itemize}

\begin{table}[h!]
  \centering
  \caption{The tally estimators supported by each MGXS type.}
  \small
  \label{tab:mgxs-tally-estimators} 
  \vspace{6pt}
  \begin{tabular}{l c c}
  \toprule
  \textbf{Class} &
  \textbf{Analog} &
  \textbf{Track-length} \\
  \midrule
  \multicolumn{3}{c}{\bf Prompt Neutron Constants} \\
  \midrule
  \texttt{AbsorptionXS} & \cmark & \cmark \\
  \texttt{CaptureXS} & \cmark & \cmark \\
  \texttt{Chi} & \cmark & \xmark \\
  \texttt{FissionXS} & \cmark & \cmark \\
  \texttt{InverseVelocity} & \cmark & \xmark \\
  \texttt{KappaFissionXS} & \cmark & \cmark \\
  \texttt{MultiplicityMatrixXS} & \cmark & \xmark \\
  \texttt{NuFissionMatrixXS} & \cmark & \xmark \\
  \texttt{ScatterXS} & \cmark & \cmark \\
  \texttt{ScatterMatrixXS} & \cmark & \xmark \\
  \texttt{ScatterProbabilityMatrixXS} & \cmark & \xmark \\
  \texttt{TotalXS} & \cmark & \xmark \\
  \texttt{TransportXS} & \cmark & \xmark \\
  \midrule
  \multicolumn{3}{c}{\bf Delayed Neutron Constants} \\
  \midrule
  \texttt{Beta} & \cmark & \xmark \\
  \texttt{ChiDelayed} & \cmark & \xmark \\
  \texttt{DelayedNuFissionXS} & \cmark & \xmark \\
  \texttt{DecayRate} & \cmark & \xmark \\
  \bottomrule
\end{tabular}
\end{table}
