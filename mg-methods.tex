%%%%%%%%%%%%%%%%%%%%%%%%%%%%%%%%%%%%%%%%%%%%%%%%%%%%%%%%%%%%%%%%%%%%%%%%%%%%%%%%
\section{Multi-Group Transport Methods}
\label{sec:mg-theory}

A key trend in recent years has been the steady progress towards full-core neutron transport-based reactor analysis tools. The standard methods used for reactor analysis today continue to be based on diffusion theory, which enables orders of magnitude computational performance improvements with respect to transport methods. However, these techniques rely on a number of assumptions and approximations which are not valid for all reactor types. Although the computational requirements for full-core transport-based simulations have precluded their widespread deployment, the continuing growth of cheap parallel processing power has made the prospects for such tools increasingly feasible.

Transport-based methods for reactor physics apply a variety of approximations to solve the critical, steady-state Boltzmann transport equation,

\begin{dmath}
\label{eqn:transport-ce}
\mathbf{\Omega} \cdot \nabla \psi(\mathbf{r},\mathbf{\Omega},E) + \Sigma_{t}(\mathbf{r},E)\psi(\mathbf{r},\mathbf{\Omega},E) = \int\displaylimits_{0}^{\infty}\int\displaylimits_{4\pi} \Sigma_{s}(\mathbf{r},{\mathbf{\Omega'}\rightarrow\mathbf{\Omega}},{E'\rightarrow E}) \psi(\mathbf{r},\mathbf{\Omega'},E') \mathrm{d}\mathbf{\Omega'} \mathrm{d}E' + \frac{1}{k_{eff}}\int\displaylimits_{0}^{\infty}\int\displaylimits_{4\pi} \nu\Sigma_{f}(\mathbf{r},{\mathbf{\Omega'}\rightarrow \mathbf{\Omega}},{E'\rightarrow E})\psi(\mathbf{r},\mathbf{\Omega'},E') \mathrm{d}\mathbf{\Omega'} \mathrm{d}E'
\end{dmath}

\noindent which is integro-differential in the neutron angular flux $\psi(\mathbf{r},\mathbf{\Omega},E)$ in space $\mathbf{r}$, direction of travel $\mathbf{\Omega}$ and energy $E$. The equation depends on the macroscopic total, scattering and fission cross sections $\Sigma_{t}$, $\Sigma_{s}$ and $\nu\Sigma_{f}$, respectively, and the eigenvalue $k_{eff}$ of the critical system.

The accurate determination of the neutron flux is primarily challenged by the complicated energy structure of the cross sections. Monte Carlo transport methods may be employed to exactly treat the energy dependence in \cref{eqn:transport-ce}\footnote{The treatment is only as exact as the uncertainties in measured nuclear cross section data will permit.}, but it is computationally burdensome and impractical for routine nuclear reactor analysis. Deterministic methods do not make use of continuous energy cross section data and instead discretize the energy domain. Deterministic methods apply a variety of approximations to \cref{eqn:transport-ce}, including spatial and energy discretization, angular expansion of the scattering kernel, and an isotropic fission source. The deterministic OpenMOC transport code \cref{eqn:transport-mg} used in this paper solves the following form of the multi-group transport equation,

\begin{equation}
\label{eqn:transport-mg}
\mathbf{\Omega} \cdot \nabla \psi_{g}(\mathbf{r},\mathbf{\Omega}) + \Sigma_{tr,g}\psi_{g}(\mathbf{r},\mathbf{\Omega}) = \frac{1}{4\pi} \sum_{g'=1}^{G} \Sigma_{s,k,g' \rightarrow g}\phi_{g'}(\mathbf{r}) + \frac{\chi_{k,g}}{4\pi k_{eff}}\sum_{g'=1}^{G} \nu\Sigma_{f,k,g'}\phi_{g'}(\mathbf{r})
\end{equation}

\noindent where the subscript $k$ corresponds to the discretized spatial mesh cell $k$ and energy group $g \in \left\{1, 2, \ldots, G\right\}$ spans a range of energies from $\left[E_{g}, E_{g-1}\right]$. This form of the equation depends on the angular- and energy-integrated scalar flux $\phi_g$. As a result of the approximations applied to transform \cref{eqn:transport-ce} to \cref{eqn:transport-mg}, the multi-group transport-corrected total cross section $\Sigma_{tr,k,g}$, isotropic group-to-group scattering matrix $\Sigma_{s,k,g'\rightarrow g}$, fission production cross section $\nu\Sigma_{f,k,g}$ and fission energy spectrum $\chi_{k,g}$ must be pre-computed and supplied as parameters to multi-group codes which solve this form of the multi-group transport equation.
