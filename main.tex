\documentclass{ansjournal}

\usepackage{microtype}                     % Improve typography
\usepackage{amsmath, mathtools}             % AMS Math extensions
\usepackage{booktabs}                      % Improved table spacing
\usepackage{siunitx}
\usepackage{breqn}
\usepackage{nicefrac}
\usepackage{multirow}
\usepackage{enumitem}
\usepackage{subcaption}
\usepackage{dblfloatfix}
\usepackage{color}
\usepackage{todonotes}
\usepackage{etoolbox}
\usepackage{listings}
\usepackage{color}
\usepackage{bbold}
\usepackage{breqn}

% Checkmarks
\usepackage{amssymb}% http://ctan.org/pkg/amssymb
\usepackage{pifont}% http://ctan.org/pkg/pifont
\newcommand{\cmark}{\ding{51}}%
\newcommand{\xmark}{\ding{55}}%

\newcommand{\code}[2]{
  \subsection*{#1}
  \lstinputlisting{#2}
}

\lstset{
  basicstyle=\footnotesize\ttfamily,
  columns=fullflexible,
  showstringspaces=false,
  commentstyle=\color{gray}\upshape,
  frame=single,
  xleftmargin=0.8in,
  xrightmargin=0.8in
}

\crefname{lstlisting}{Listing}{Listings}

% Patch to make equation autoref counter work
\makeatletter
\patchcmd\eq@setnumber{\stepcounter}{\refstepcounter}{}{%
  \errmessage{Patching \noexpand\eq@setnumber failed}%
}
\makeatother

\hypersetup{colorlinks=true,
  pdftitle={Multi-Group Cross Section Generation with the OpenMC Monte Carlo Particle Transport Code},
  pdfauthor={William Boyd, Adam Nelson, Paul Romano, Samuel Shaner, Benoit Forget and Kord Smith}}

\title{Multi-Group Cross Section Generation with the OpenMC Monte Carlo Particle Transport Code}

\author[a]{William Boyd\footnote{Email: \href{mailto:boyd.william.r@gmail.com}{boyd.william.r@gmail.com}}}
\author[b]{Adam Nelson}
\author[c]{Paul K. Romano}
\author[d]{Samuel Shaner\footnote{This author was a graduate research assistant at the Massachusetts Institute of Technology at the time this research was conducted.}}
\author[a]{Benoit Forget}
\author[a]{Kord Smith}

\affil[a]{Massachusetts Institute of Technology, Department of Nuclear Science
  and Engineering, 77 Massachusetts Avenue, Building 24, Cambridge,
  Massachusetts 02139}
\affil[b]{Naval Reactors, 1240 Isaac Hull Ave. SE, Washington Navy Yard, D.C. 20376}
\affil[c]{Argonne National Laboratory, Mathematics and Computer Science
  Division, 9700 South Cass Avenue, Lemont, Illinois 60439}
\affil[d]{Yellowstone Energy, 11631 Lanesborough Way, Farragut, Tennessee 37934}

%%%%%%%%%%%%%%%%%%%%%%%%%%%%%%%%%%%%%%%%%%%%%%%%%%%%%%%%%%%%%%%%%%%%%%%%%%%%%%%
\abstractText{High-fidelity deterministic transport codes require highly accurate multi-group cross sections (MGXS). Monte Carlo is increasingly cited as a reactor-agnostic approach to MGXS generation since it is unconstrained by the engineering-based approximations that limit the applicability of deterministic MGXS generation tools. This paper introduces a new framework that uses the OpenMC Monte Carlo code to generate MGXS for use in multi-group transport codes. The \texttt{openmc.mgxs} module is built atop OpenMC's Python API to process tally data output by the OpenMC executable. This paper validates the module to generate MGXS that enable the multi-group OpenMOC transport code to compute eigenvalues to within 50 pcm and fission rates to within 1\% of reference solutions for two heterogeneous pressurized water reactor benchmarks.}

%%%%%%%%%%%%%%%%%%%%%%%%%%%%%%%%%%%%%%%%%%%%%%%%%%%%%%%%%%%%%%%%%%%%%%%%%%%%%%%

\keywordsText{Monte Carlo neutron transport, multi-group cross section generation, OpenMC}

\begin{document}

\maketitle

%%%%%%%%%%%%%%%%%%%%%%%%%%%%%%%%%%%%%%%%%%%%%%%%%%%%%%%%%%%%%%%%%%%%%%%%%%%%%%%
\section{Introduction}
\label{sec:intro}
%%%%%%%%%%%%%%%%%%%%%%%%%%%%%%%%%%%%%%%%%%%%%%%%%%%%%%%%%%%%%%%%%%%%%%%%%%%%%%%

The last two decades have seen growing interest in Monte Carlo (MC) as a means to generate multi-group cross section (MGXS) libraries. Most MC-based MGXS generation schemes to date focus on generating few-group constants for coarse mesh diffusion codes. These schemes aim to improve the accuracy of standard diffusion codes for analysis of atypical core configurations for which the simplifications made by multi-level deterministic MGXS generation methods are not necessarily applicable. These efforts replace the separate resonance self-shielding and deterministic lattice physics calculation steps in multi-step approaches with fully-detailed MC calculations of each assembly to compute the few-group constants needed by whole core diffusion codes. The widely used Serpent code\cite{leppanen2015serpent} has led this trend over the last decade, and a few authors have applied the MCNP\cite{pounders2006stochastically} and McCARD\cite{shim2008generation} codes in a similar fashion. This paper presents new capabilities introduced in the OpenMC\cite{romano2015openmc} particle transport code for multi-group cross section generation for fine-mesh multi-group neutron transport applications.

This paper is organized as follows. \cref{sec:mg-theory} summarizes the key multi-group constants required by deterministic multi-group codes. \cref{sec:mgxs-mc} details the mathematical formulation for stochastic integration of each of the multi-group constants presently computable by OpenMC. \cref{sec:openmc} highlights the core OpenMC features which provide the foundation for the \texttt{openmc.mgxs} module for MGXS generation introduced in \cref{sec:design}. The OpenMOC\cite{boyd2014openmoc} multi-group transport code is used to validate the MGXS generated by OpenMC in \cref{sec:validate}.

%%%%%%%%%%%%%%%%%%%%%%%%%%%%%%%%%%%%%%%%%%%%%%%%%%%%%%%%%%%%%%%%%%%%%%%%%%%%%%%%
\section{Multigroup Transport Methods}
\label{sec:mg-theory}

A key trend in recent years has been the steady progress toward full-core neutron transport-based reactor analysis tools. Transport-based methods for reactor physics apply a variety of approximations to solve the following form of the steady-state Boltzmann transport equation,

\begin{dmath}
\label{eqn:transport-ce}
\mathbf{\Omega} \cdot \nabla \psi(\mathbf{r},\mathbf{\Omega},E) + \Sigma_{t}(\mathbf{r},E)\psi(\mathbf{r},\mathbf{\Omega},E) = \int\displaylimits_{0}^{\infty}\int\displaylimits_{4\pi} \Sigma_{s}(\mathbf{r},{\mathbf{\Omega'}\rightarrow\mathbf{\Omega}},{E'\rightarrow E}) \psi(\mathbf{r},\mathbf{\Omega'},E') \mathrm{d}\mathbf{\Omega'} \mathrm{d}E' + \frac{1}{k_{\textrm{eff}}}\int\displaylimits_{0}^{\infty}\int\displaylimits_{4\pi} \nu\Sigma_{f}(\mathbf{r},{\mathbf{\Omega'}\rightarrow \mathbf{\Omega}},{E'\rightarrow E})\psi(\mathbf{r},\mathbf{\Omega'},E') \mathrm{d}\mathbf{\Omega'} \mathrm{d}E'\,,
\end{dmath}

\noindent which is integro-differential in the neutron angular flux $\psi(\mathbf{r},\mathbf{\Omega},E)$ in space $\mathbf{r}$, direction of travel $\mathbf{\Omega}$, and energy $E$. The equation depends on the macroscopic total, scattering, and fission production cross sections $\Sigma_{t}$, $\Sigma_{s}$, and $\nu\Sigma_{f}$, respectively, and the eigenvalue $k_{\textrm{eff}}$ of the critical system.

The accurate determination of the neutron flux is challenged primarily by the complicated energy structure of the cross sections. Monte Carlo transport methods are able to exactly treat the energy dependence in \cref{eqn:transport-ce}\footnote{The treatment is only as exact as the uncertainties in measured nuclear cross section data will permit.}, but they are computationally burdensome and impractical for routine nuclear reactor analysis. Deterministic methods do not make use of continuous energy cross sections and instead rely on multigroup constants as a result of several simplifying approximations to \cref{eqn:transport-ce}, including energy and spatial discretization, angular expansion of the scattering kernel, and an isotropic fission source. The deterministic OpenMOC transport code \textcolor{red}{(see \cref{subsec:openmoc})} used in this paper solves the following form of the multigroup transport equation,

\begin{equation}
\label{eqn:transport-mg}
\mathbf{\Omega} \cdot \nabla \psi_{g}(\mathbf{r},\mathbf{\Omega}) + \Sigma_{tr,g}\psi_{g}(\mathbf{r},\mathbf{\Omega}) = \frac{1}{4\pi} \sum_{g'=1}^{G} \Sigma_{s,k,g' \rightarrow g}\phi_{g'}(\mathbf{r}) + \frac{\chi_{k,g}}{4\pi k_{\textrm{eff}}}\sum_{g'=1}^{G} \nu\Sigma_{f,k,g'}\phi_{g'}(\mathbf{r})\,,
\end{equation}

\noindent where the subscript $k$ corresponds to the discretized spatial mesh cell $k$ and energy group $g \in \left\{1, 2, \ldots, G\right\}$ spans a range of energies from $\left[E_{g}, E_{g-1}\right]$. This form of the equation depends on the angular- and energy-integrated scalar flux $\phi_g$. The multigroup transport-corrected total cross section $\Sigma_{tr,k,g}$, isotropic group-to-group scattering matrix $\Sigma_{s,k,g'\rightarrow g}$, fission production cross section $\nu\Sigma_{f,k,g}$ and fission energy spectrum $\chi_{k,g}$ must be precomputed and supplied as parameters to multigroup codes that solve this form of the multigroup transport equation.

%%%%%%%%%%%%%%%%%%%%%%%%%%%%%%%%%%%%%%%%%%%%%%%%%%%%%%%%%%%%%%%%%%%%%%%%%%%%%%%
\section{MGXS Generation with Monte Carlo}
\label{sec:mgxs-mc}
%%%%%%%%%%%%%%%%%%%%%%%%%%%%%%%%%%%%%%%%%%%%%%%%%%%%%%%%%%%%%%%%%%%%%%%%%%%%%%%


%%%%%%%%%%%%%%%%%%%%%%%%%%%%%%%%%%%%%%%%%%%%%%%%%%%
\subsection{Tally Types Needed for MGXS Generation}
\label{subsec:tally-types}

This section outlines the types of tallies needed to compute MGXS. 

\begin{table}[h!]
  \centering
  \caption[Tally types for MGXS generation]{The types of tallies used in MGXS generation with OpenMC.}
  \scriptsize
  \label{table:chap3-tally-types}
  \vspace{6pt}
  \begin{tabular}{ m{1.3cm} m{1cm} m{2cm} m{2.5cm} m{2.5cm} m{1.5cm} }
  \toprule
  {\bf Name} &
  {\bf Symbol} &
  {\bf Tally} &
  {\bf Score} &
  {\bf Filters} &
  {\bf Estimator} \\

  \specialrule{.2em}{.1em}{.1em}

  \multirow{2}{*}[-0.7em]{\bf General} & \multirow{2}{*}[-0.7em]{$\Sigma_{x,k,g}$} & $\langle \Sigma_{x}, \psi \rangle_{k,g}$ & reaction $x$ & \parbox{2cm}{\texttt{material}/\texttt{cell} \texttt{energy}} & \texttt{track-length} \\
  \cline{3-6}
  & & $\langle \psi \rangle_{k,g}$ & {\texttt{flux}} & \parbox{2cm}{\texttt{material}/\texttt{cell} \texttt{energy}} & \texttt{track-length} \\

  \specialrule{.2em}{.1em}{.1em}

  \multirow{2}{*}[-0.7em]{\bf Total} & \multirow{2}{*}[-0.7em]{$\Sigma_{t,k,g}$} & $\langle \Sigma_{t}, \psi \rangle_{k,g}$ & \texttt{total} & \parbox{2cm}{\texttt{material}/\texttt{cell} \texttt{energy}} & \texttt{track-length} \\
  \cline{3-6}
  & & $\langle \psi \rangle_{k,g}$ & \texttt{flux} & \parbox{2cm}{\texttt{material}/\texttt{cell} \texttt{energy}} & \texttt{track-length} \\

  \specialrule{.2em}{.1em}{.1em}

  \multirow{3}{*}[-1em]{\parbox{1.5cm}{\bf Transport-Corrected Total}} & \multirow{3}{*}[-1em]{$\hat{\Sigma}_{tr,k,g}$} & $\langle \Sigma_{t}, \psi \rangle_{k,g}$ & \texttt{total} & \parbox{2cm}{\texttt{material}/\texttt{cell} \texttt{energy}} & \texttt{analog} \\
  \cline{3-6}
  & & $\langle \Sigma_{s1}, \psi \rangle_{k,g'\rightarrow g}$ & \texttt{nu-scatter-1} & \parbox{2cm}{\texttt{material}/\texttt{cell} \texttt{energyout}} & \texttt{analog} \\
  \cline{3-6}
  & & $\langle \psi \rangle_{k,g}$ & \texttt{flux} & \parbox{2cm}{ \texttt{material}/\texttt{cell} \texttt{energy}} & \texttt{analog} \\

  \specialrule{.2em}{.1em}{.1em}

  \multirow{2}{*}[-0.5em]{\parbox{1.5cm}{\bf Scattering Matrix}} & \multirow{2}{*}[-0.5em]{$\Sigma_{s,k,g'\rightarrow g}$} & $\langle \Sigma_{s0}, \psi \rangle_{k,g'\rightarrow g}$ & \texttt{nu-scatter-0} & \parbox{2cm}{\texttt{material}/\texttt{cell} \texttt{energy} \texttt{energyout}} & \texttt{analog} \\
  \cline{3-6}
  & & $\langle \psi \rangle_{k,g}$ & \texttt{flux} & \parbox{2cm}{\texttt{material}/\texttt{cell} \texttt{energy}} & \texttt{analog} \\

  \specialrule{.2em}{.2em}{.2em}

  \multirow{3}{*}[-1em]{\parbox{1.5cm}{\bf Transport-Corrected Scattering Matrix}} & \multirow{3}{*}[-1em]{$\Sigma_{s,k,g'\rightarrow g}$} & $\langle \Sigma_{s0}, \psi \rangle_{k,g'\rightarrow g}$ & \texttt{nu-scatter-0} & \parbox{2cm}{\texttt{material}/\texttt{cell} \texttt{energy} \texttt{energyout}} & \texttt{analog} \\
  \cline{3-6}
  & & $\langle \Sigma_{s1}, \psi \rangle_{k,g'\rightarrow g}$ & \texttt{nu-scatter-1} & \parbox{2cm}{\texttt{material}/\texttt{cell} \texttt{energyout}} & \texttt{analog} \\
  \cline{3-6}
  & & $\langle \psi \rangle_{k,g}$ & \texttt{flux} & \parbox{2cm}{\texttt{material}/\texttt{cell} \texttt{energy}} & \texttt{analog} \\

  \specialrule{.2em}{.1em}{.1em}

  \multirow{2}{*}[-0.5em]{\parbox{1.5cm}{\bf Fission \hspace{1cm} Production}} & \multirow{2}{*}[-0.5em]{$\nu\Sigma_{f,k,g}$} & $\langle \nu\Sigma_{f}, \psi \rangle_{k,g}$ & \texttt{nu-fission} & \parbox{2cm}{\texttt{material}/\texttt{cell} \texttt{energy}} & \texttt{track-length} \\
  \cline{3-6}
  & & $\langle \psi \rangle_{k,g}$ & \texttt{flux} & \parbox{2cm}{\texttt{material}/\texttt{cell} \texttt{energy}} & \texttt{track-length} \\

  \specialrule{.2em}{.1em}{.1em}
  
  \parbox{1.5cm}{\bf Fission Spectrum} & $\chi_{k,g}$ & $\langle \nu\Sigma_{f}, \psi \rangle_{k,g'\rightarrow g}$ & \texttt{nu-fission} & \parbox{2cm}{\texttt{material}/\texttt{cell} \texttt{energy} \texttt{energyout}} & \texttt{analog} \\
  \midrule

\end{tabular}
\end{table}
%%%%%%%%%%%%%%%%%%%%%%%%%%%%%%%%%%%%%%%%%%%%%%%%%%%%%%%%%%%%%%%%%%%%%%%%%%%%%%%
\section{OpenMC Core Features}
\label{sec:openmc}
%%%%%%%%%%%%%%%%%%%%%%%%%%%%%%%%%%%%%%%%%%%%%%%%%%%%%%%%%%%%%%%%%%%%%%%%%%%%%%%

The preceding section outlined the procedures for estimating multi-group cross sections based on flux and reaction rate tallies from a Monte Carlo code. It is conceptually simple to use a Monte Carlo code to compute these tallies and report macroscopic or microscopic multi-group cross sections as part of the output of the code. Indeed, this is the approach adopted by the Serpent Monte Carlo code\cite{leppanen2016homog}. The MGXS generation framework implemented in OpenMC takes a fundamentally different and lightweight alternative approach -- instead of using the transport kernel to directly compute and report MGXS, the standard output of the code only includes flux and reaction rate values. After a simulation is complete, the MGXS are calculated as a post-processing task on the reported tallies by a Python Application Programming Interface (API).

OpenMC is an open source Monte Carlo particle transport code which is primarily intended for use in neutron criticality calculations. It is capable of simulating three-dimensional models using constructive solid geometry. It also supports both continuous-energy and multi-group cross section data in a native HDF5\cite{koranne2011hdf5} data format\cite{romano2017epjwoc} that can be generated from ACE files. The following sections summarize OpenMC's tally system and Python API which form the foundation for the MGXS generation module presented in \cref{sec:design}.

%%%%%%%%%%%%%%%%%%%%%%%%
\subsection{Tally System}
\label{subsec:tallies}

OpenMC features a flexible, low-overhead tally system that enables users to obtain physical results of interest. Tallies are defined by combinations of \emph{filters} and \emph{scores}. Each filter limits what events can score to the tally based on the phase space variables. For example, a filter could limit scoring events to particles traveling within a specified cell or a specified range of pre-collision energies. Each score identifies a physical quantity to be scored when an event occurs that matches the specified filters. Looking at \cref{eqn:inner-prod}, filters correspond to the limits of integration and scores correspond to the integrand itself:
\begin{equation}
\langle \Sigma_x, \psi \rangle = \underbrace{\int_{V} \int_{S} \int_{E}}_{\text{filters}} \underbrace{\Sigma_{x}(\mathbf{r},E)\psi(\mathbf{r},E,\mathbf{\Omega})}_{\text{scores}} \mathrm{d}E\mathrm{d}\mathbf{\Omega}\mathrm{d}\mathbf{r}
\end{equation}
In addition to filters and scores, it is also possible to obtain reaction rates for individual nuclides by specifying a list of nuclides. Each tally definition is permitted to have multiple filters, multiple scores, and multiple nuclides. Additionally, the estimator used to score events can also be specified on a per-tally basis.

A wide range of filters and scores have been implemented as of the version 0.9.0 release of OpenMC\cite{openmc-090}. The available filters can generally be categorized as follows:
\begin{itemize}[noitemsep]
\item \emph{Spatial domain}: Tally events by universe, material, cell, mesh
\item \emph{Energy domain}: Tally events based on both incoming and outgoing particle energy
\item \emph{Angle domain}: Tally events based on a particle's polar and azimuthal direction or scattering angle
\item \emph{Energy function}: Multiplies tally scores by an arbitrary function of incident energy
\item \emph{Delayed group}: Tally events which produced neutrons in particular delayed groups
\end{itemize}
The available scores in version 0.9.0 include: particle flux, all invidual reaction rates, neutron production from fission (total, prompt, or delayed), Legendre and spherical harmonic scattering moments, spherical harmonic flux moments, inverse velocity, recoverable fission energy release, and the delayed neutron production-weighted decay rate. Taken together, these filters and scores permit all inner products identified in \cref{tab:tally-types} to be estimated via tallies in OpenMC.

%%%%%%%%%%%%%%%%%%%%%%%
\subsection{Python API}
\label{subsec:pyapi}

A fully-featured Python API enables programmatic pre- and post-processing for OpenMC\cite{boyd2016bigdata}. The API makes it possible to write a single dynamic Python script to specify the simulation parameters, execute the simulation and analyze the resultant tally dataset. In addition, the API makes it possible to leverage the extensive ecosystem of Python packages for scientific computing alongside OpenMC in a simulation workflow. OpenMC's dynamic object-oriented data processing model -- fusing the geometry and materials configuration with tallied data -- enables the rapid calculation, indexing, and storage of MGXS from tallies over specified spatial domains. This section describes two features developed for the API to support OpenMC's MGXS generation module.


%%%%%%%%%%%%%%%%%%%%%%%%%%%%%%%%
\subsubsection{Tally Slicing and Merging}
\label{subsubsec:tally-slice-merge}

Two useful and related features in the OpenMC Python API for MGXS generation are \emph{tally merging} and \emph{tally slicing} as depicted in \cref{fig:tally-merge-slice}. It is intuitively useful to create separate \texttt{Tally} objects for each spatial domain and reaction type when generating the OpenMC inputs necessary to compute MGXS. However, this necessarily leads to a large number (10$^2$ -- 10$^3$) of distinct tally objects for large, complex geometries, which poses a computational bottleneck since the overhead to tally in OpenMC scales as $\mathcal{O}(N)$ for $N$ tallies\footnote{Note that $N$ refers to the number of tally definitions, each of which can contain multiple filters, nuclides, and scores.}. To compensate for this, the Python API's \texttt{Tally} class automatically merges user-specified tallies for input generation. Similarly, the API supports the slicing of tallies to simplify downstream data processing which may comprise energy-, nuclide-, and/or reaction-dependent transformations of the tally data. 

\begin{figure}
\begin{subfigure}{\textwidth}
  \centering
  \includegraphics[width=0.6\linewidth]{figures/tally-merge}
  \caption{}
\end{subfigure}
\begin{subfigure}{\textwidth}
  \centering
  \includegraphics[width=0.6\linewidth]{figures/tally-slice}
  \caption{}
\end{subfigure}
\caption{Two \texttt{Tally} objects for different spatial volumes are merged into a single \texttt{Tally} (a). A single \texttt{Tally} is sliced by spatial volume into two distinct \texttt{Tally} objects (b).}
\label{fig:tally-merge-slice}
\end{figure}

%%%%%%%%%%%%%%%%%%%%%%%%%%%%%%%%
\subsubsection{Tally Arithmetic}
\label{subsubsec:tally-arithmetic}

A variety of reaction rate and flux tallies must be arithmetically combined in order to compute MGXS with Monte Carlo. At the most general level, a reaction rate tally must be divided by a flux tally for each energy group, nuclide and tally volume. The Python API provides a novel feature known as \emph{tally arithmetic} to enable arithmetic combinations of tallies with efficient vectorized numerical operations across energy groups, nuclides and spatial tally zones.

Tally arithmetic is an object-oriented data processing feature which arithmetically combines two or more tallies and/or scalar values into new \emph{derived tallies}. The Python API overloads the \texttt{Tally} class' operators for addition, subtraction, multiplication and division. Furthermore, the \texttt{Tally} class supports summation and averaging operations across some or all of its filter, nuclide or score bins.

Multi-group cross sections may be simply and efficiently computed with tally arithmetic. For example, the following code snippet illustrates how tally slicing and arithmetic are used to compute a total MGXS. The total MGXS that is returned from the tally division operation is encapsulated within a \texttt{Tally} object. This is the approach used by the MGXS generation module created for OpenMC.

\lstinputlisting[language=Python, basicstyle=\ttfamily\scriptsize, caption={MGXS calculation with tally arithmetic.}, label={lst:python-input}]{listings/tally-arithmetic.py}

Tally arithmetic automatically propagates the uncertainties of the tally operands through the arithmetic operation to estimate the uncertainty of the resulting derived tally. Estimates of the variance for derived tallies are deduced from standard error propagation theory~\cite{bevington2003data}. The division operator is primarily used to to compute MGXS from MC tallies. Consider two random variables $X$ and $Y$, generated from distributions with variances $\sigma_{X}^2$ and $\sigma_{Y}^2$ which are divided into a new random variable $Z$ with variance $\sigma_{Z}^2$:

\begin{equation}
\label{eqn:div-prop}
\sigma_{Z}^{2} \approx Z^{2}\left[\left(\frac{\sigma_{X}}{X}\right)^{2} + \left(\frac{\sigma_{Y}}{Y}\right)^{2} - 2\frac{\sigma_{XY}}{Z}\right]
\end{equation}

\noindent The variables $X$ and $Y$ may correspond to reaction rates and the flux tallies, while $Z$ could correspond to the MGXS.

The covariance $\sigma_{XY}$ is not generally computable using the standard formulation for a tally estimator in a Monte Carlo simulation. Although it would be possible to estimate the covariance using ensemble statistics, this is typically infeasible. Instead, the covariance term in \cref{eqn:div-prop} is currently neglected by OpenMC's implementation of tally arithmetic. In general, the random variables for reaction rates and fluxes in the same volume of phase space are highly correlated, such that a conservative estimate of the variance for MGXS is obtained by neglecting the covariance.

%%%%%%%%%%%%%%%%%%%%%%%%%%%%%%%%%%%%%
\subsection{Distributed Cell Tallies}
\label{subsec:distribcells}

%Many Monte Carlo codes, including OpenMC, use some variant of combinatorial geometry (CG) because it can represent arbitrary, repeating geometries such as fuel pins and assemblies. However, the CG approach is challenged by applications which require tallies in each instance of a repeated cell throughout a reactor geometry. The ``brute force'' solution is to instantiate a unique cell for each distinct tally zone. However, this defeats the purpose of using CG for its compact representation, and it is not scalable to problems with large tally datasets such as those considered in this thesis.

The \emph{distributed cell tally} algorithm was implemented in OpenMC \cite{lax2014distribcell} to permit simply defined spatial tally zones across repeated cell instances. The distribcell tally algorithm -- abbreviated as the \emph{distribcell} algorithm -- may be used to compute spatially-varying MGXS across fuel pin cell instances. The distribcell algorithm classifies each unique cell instance using a process which consumes orders of magnitude less memory than would be required by the ``brute force'' approach necessary to accomplish the same objective with other commonly used Monte Carlo codes. Only a single transparent line of XML input is necessary to define a distribcell tally which may span across an arbitrary number of instances for a particular cell. Furthermore, the Python API may be used to perform efficient vectorized transformations of distribcell tally data stored as contiguous NumPy arrays to compute MGXS.

%%%%%%%%%%%%%%%%%%%%%%%%%%%%%%%%%%%%%%%%%%%%%%%%%%%%%%%%%%%%%%%%%%%%%%%%%%%%%%%
\section{MGXS Module Design}
\label{sec:design}
%%%%%%%%%%%%%%%%%%%%%%%%%%%%%%%%%%%%%%%%%%%%%%%%%%%%%%%%%%%%%%%%%%%%%%%%%%%%%%%

The OpenMC Python API features a module called \texttt{openmc.mgxs} that was designed to generate multi-group cross sections. The \texttt{openmc.mgxs} module is built atop the underlying core features in the rest of the API to support a seamless interface for both input generation and downstream data processing of MGXS from Python. In particular, one may specify the MGXS to compute and the \texttt{openmc.mgxs} module will construct the necessary \texttt{Tally} objects. The \texttt{Tally} objects may be easily exported to XML input files and used to containerize and process the tally data produced by an OpenMC simulation. The \texttt{openmc.mgxs} module thereby leverages the existing classes and features (\textit{e.g.}, tally arithmetic and Pandas DataFrames) provided by the Python API.

The \texttt{openmc.mgxs} module uses an object-oriented design based on an abstract \texttt{MGXS} class with subclasses for different cross section types. The \texttt{MGXS} subclasses, as listed in \cref{tab:mgxs-types}, compute macroscopic or microscopic multi-group constants in one or more arbitrary energy group structures from MC tallies. The \texttt{openmc.mgxs} module also includes a \texttt{Library} class which automates the construction of \texttt{MGXS} objects for different group structures, spatial domains, and reaction types.

\begin{table}[h!]
  \centering
  \caption{The multi-group cross section types implemented by the \texttt{openmc.mgxs} module.}
  \small
  \label{tab:mgxs-types}
  \vspace{6pt}
  \begin{tabular}{l l}
  \toprule
  \textbf{Class} &
  \textbf{Description} \\
  \midrule
  \multicolumn{2}{c}{\bf Prompt Neutron Constants} \\
  \midrule
  \texttt{AbsorptionXS} & Absorption \\
  \texttt{CaptureXS} & Radiative capture \\
  \texttt{Chi} & Fission emission spectrum \\
  \texttt{FissionXS} & Fission \\
  \texttt{InverseVelocity} & Inverse neutron velocity \\
  \texttt{KappaFissionXS} & Fission energy release \\
  \texttt{MultiplicityMatrixXS} & Scattering multiplicity matrix \\
  \texttt{NuFissionMatrixXS} & Fission production matrix \\
  \texttt{ScatterXS} & Scattering \\
  \texttt{ScatterMatrixXS} & Scattering matrix \\
  \texttt{ScatterProbabilityMatrixXS} & Scattering probability matrix \\
  \texttt{TotalXS} & Total collision \\
  \texttt{TransportXS} & Transport-corrected total collision \\
  \midrule
  \multicolumn{2}{c}{\bf Delayed Neutron Constants} \\
  \midrule
  \texttt{Beta} & Delayed neutron fraction \\
  \texttt{ChiDelayed} & Delayed fission spectrum \\
  \texttt{DecayRate} & Delayed neutron precursors decay rate \\
  \texttt{DelayedNuFissionXS} & Fission delayed neutron production \\
  \texttt{DelayedNuFissionMatrixXS} & Fission delayed neutron production matrix \\
  \bottomrule
\end{tabular}
\end{table}


%%%%%%%%%%%%%%%%%%%%%%%%%%%%%
\subsection{Workflow}
\label{subsec:workflow}

The MGXS generation workflow begins by creating instances of \texttt{MGXS} subclasses. In general, it is assumed that the model geometry and materials have already been defined. There are two options for instantiating \texttt{MGXS} objects: (1) manual instantiation of the subclasses or (2) automated instantiation via the \texttt{Library} class. In the first option, an instance of an \texttt{MGXS} subclass is assigned a spatial domain and an energy-group structure. One can also specify whether cross sections are desired for all nuclides or on a per-nuclide basis through the \texttt{MGXS.by_nuclide} attribute. Tallies are produced automatically via a \texttt{MGXS.tallies} attribute and can be appended to a collection of tallies represented by a \texttt{Tallies} class object. The tallies collection in turn has an \texttt{export_to_xml()} method which writes out the XML input file to be read by the transport solver. Alternatively, the \texttt{Library} class allows one to specify that multiple cross sections should be computed for multiple spatial domains and a given energy-group structure. The \texttt{Library.add_to_tallies_file()} method then adds all necessary tallies to compute the MGXS to an existing tallies collection, merging together any tallies that can be merged to reduce the total number of tallies.

After a simulation has completed, an HDF5 \emph{state point} file is written
that contains the tally results. From the Python API, the \texttt{StatePoint}
object reads in the tally results and can then be used by the
\texttt{MGXS.load_from_statepoint()} method which loads those results into an
\texttt{MGXS} object. At that point, the multi-group cross sections can be displayed on standard output, saved to a file, or converted to a Pandas DataFrame, as described further below. Two examples of the workflow for generating MGXS manually and using the \texttt{Library} are shown in \cref{lst:mgxs-manual} and \cref{lst:mgxs-library}, respectively.

\lstinputlisting[language=Python, basicstyle=\ttfamily\scriptsize, caption={MGXS calculation with manual object instantiation.}, label={lst:mgxs-manual}]{listings/mgxs-manual.py}

\lstinputlisting[language=Python, basicstyle=\ttfamily\scriptsize, caption={MGXS calculation with \texttt{Library}-automated object instantiation.}, label={lst:mgxs-library}]{listings/mgxs-library.py}


%%%%%%%%%%%%%%%%%%%%%%%%%%%%%
\subsection{Data Processing and Storage}
\label{subsec:data-processing}

The \texttt{openmc.mgxs} module takes advantage of many of the existing features of the Python API discussed in \cref{sec:openmc}. When a user loads tally results from a statepoint and displays them or stores them to disk, the following sequence of events happens:

\begin{itemize}[noitemsep]
\item Reaction rate and flux \texttt{Tally} objects are obtained using tally slicing. This is necessary because during input generation, the tallies for multiple \texttt{MGXS} objects may have been merged into fewer tallies. Slicing returns a \texttt{Tally} object with only the filters, nuclides and scores required for the desired MGXS.
\item A derived cross section \texttt{Tally} object is produced by using tally arithmetic to divide the reaction rate tally by the flux tally for all energy groups, nuclides and spatial homogenization zones. The time to perform these operations is minimal since the tally results are stored in dense NumPy arrays which support vector arithmetic.
\item If a user requests a Pandas DataFrame for an \texttt{MGXS} object, it is produced by first generating a Pandas Dataframe for the derived cross section tally and then modifying it with a few user-friendly customizations for MGXS applications (\textit{e.g.}, labeling energy groups).
\end{itemize}

The \texttt{openmc.mgxs} module was developed with general design principles to generate MGXS for any multi-group neutron transport code. Although the module does not explicitly support any multi-group codes, it can export MGXS data to a variety of data storage formats, including Comma-Separated Values (CSV) and HDF5. The exported MGXS files may be easily transformed into the database or input files required by a particular multi-group code. Additionally, a cross section library can be exported to the format used by OpenMC's multi-group transport mode.

%%%%%%%%%%%%%%%%%%%%%%%%%%%%%%%%%%%%%%%%%%%%%%%%%%%%%%%%%%%%%%%%%%%%%%%%%%%%%%%
\subsection{Additional Features}
\label{sec:features}
%%%%%%%%%%%%%%%%%%%%%%%%%%%%%%%%%%%%%%%%%%%%%%%%%%%%%%%%%%%%%%%%%%%%%%%%%%%%%%%


%%%%%%%%%%%%%%%%%%%%%%%%%%%%%%%%%%%%%%%%%%%%
\subsubsection{Energy Condensation}
\label{subsec:energy-condense}

The \texttt{openmc.mgxs} module supports energy condensation in downstream data processing which is useful for exploring approximation bias in various energy group structures. For example, MGXS may be computed in some ``fine'' energy group structure and the tally data subsequently condensed to some coarser group structure \texttt{coarse_groups} for multi-group calculations with the \texttt{MGXS.get_condensed_xs(coarse_groups)} Python method. Energy condensation may be performed to arbitrarily defined coarse group structures provided the coarse group boundaries coincide with boundaries in the fine group structure.


%%%%%%%%%%%%%%%%%%%%%%%%%%%%%%%%%%%%%%%%%%%%
\subsubsection{Spatial Homogenization}
\label{subsec:pinwise-homogenize}

The \texttt{openmc.mgxs} module is designed to perform spatial homogenization on heterogeneous tally meshes for fine-mesh transport codes. In OpenMC parlance, MGXS may be computed for material, cell or universe spatial domains. In addition, the module supports MGXS calculations for repeated cell instances using distribcell spatial tally domains\cite{lax2014distribcell}. Spatial homogenization across some subset of distributed cell instances \texttt{cell_instances} can be performed using the \texttt{MGXS.get_subdomain_avg_xs(cell_instances)} Python class method.

%The \texttt{openmc.mgxs} module may also perform spatial homogenization on structured Cartesian tally meshes for coarse mesh multi-group calculations..


%%%%%%%%%%%%%%%%%%%%%%%%%%%%%%%%%%%%%%%%%%%%
\subsubsection{Microscopic MGXS}
\label{subsec:micro-macro}

When tally results are loaded into an \texttt{MGXS} object, the reaction rate tallies contain macroscopic quantities. Thus, by default, when a derived cross section tally is computed by dividing reaction rates by fluxes, a macroscopic cross section is obtained. However, all methods that output MGXS data (\textit{e.g.}, \texttt{MGXS.print_xs()}, \texttt{MGXS.get_pandas_dataframe()}, \texttt{MGXS.export_xs_data()}), also have an argument that allows the user to specify that microscopic cross sections should be calculated. This option works both for total material cross sections as well as per-nuclide cross sections.

Microscopic cross sections are computed for spatial domains with homogeneous material composition by dividing by the known isotopic number densities. Furthermore, microscopic cross sections can be calculated for spatial domains with heterogeneous material composition by using OpenMC's stochastic volume calculation mode. The stochastic volume calculation defines a bounding box for a spatial domain of interest, samples points uniformly within the bounding box, and counts how many points lie within the spatial domain, thereby estimating the volume fraction of the domain within the bounding box. Such a capability can be useful when microscopic MGXS are desired over an entire assembly or other spatial domain of similar complexity.

%%%%%%%%%%%%%%%%%%%%%%%%%%%%%%%%%%%%%%%%
\subsubsection{Isotropic in Lab Scattering}
\label{subsec:iso-in-lab}

A unique option for isotropic in the laboratory frame scattering is implemented in the OpenMC code. This feature, abbreviated as \emph{iso-in-lab}, may be useful to quantify the ability of multi-group codes to capture anisotropic scattering effects with higher order scattering matrices or transport correction schemes. The iso-in-lab feature is implemented as an optional attribute for each nuclide or element in a simulation. When iso-in-lab scattering is specified for a nuclide or element, the outgoing neutron energy is sampled from the scattering laws prescribed by the continuous energy cross section library, but the outgoing neutron direction of motion is sampled from a distribution that is isotropic in the laboratory frame.

%%%%%%%%%%%%%%%%%%%%%%%%%%%%%
\subsection{Tally Estimators}
\label{subsec:tally-est}

The \texttt{openmc.mgxs} module supports a variety of different tally estimators for each MGXS type as shown in \cref{tab:mgxs-tally-estimators}. Tallies with \textit{analog} estimators are only incremented when the scoring function of interest (\textit{i.e.}, reaction type) is explicitly sampled. In contrast, tallies with \textit{collision} and \textit{track-length} estimators are incremented for all scoring functions every time the appropriate region of phase space is sampled (\textit{e.g.}, energy group) in order to improve sample statistics. All MGX types currently support analog estimators, and most support collision and track-length estimators. However, all MGXS types which require a tally over outgoing energy (\textit{e.g.}, scattering matrix, fission spectrum, etc.) currently only support analog estimators. Future development of the \texttt{openmc.mgxs} module may implement a methodology\cite{nelson2014improved} which permits track-length estimators for tallies which depend on the outgoing neutron energy.

\begin{table}[h!]
  \centering
  \caption{The tally estimators supported by each MGXS type.}
  \small
  \label{tab:mgxs-tally-estimators}
  \vspace{6pt}
  \begin{tabular}{l c c c}
  \toprule
  \textbf{Class} &
  \textbf{Analog} &
  \textbf{Collision} &
  \textbf{Track-length} \\
  \midrule
  \multicolumn{4}{c}{\bf Prompt Neutron Constants} \\
  \midrule
  \texttt{AbsorptionXS} & \cmark & \cmark & \cmark \\
  \texttt{CaptureXS} & \cmark & \cmark & \cmark \\
  \texttt{Chi} & \cmark & & \\
  \texttt{FissionXS} & \cmark & \cmark & \cmark \\
  \texttt{InverseVelocity} & \cmark & \cmark & \cmark \\
  \texttt{KappaFissionXS} & \cmark & \cmark & \cmark \\
  \texttt{MultiplicityMatrixXS} & \cmark & & \\
  \texttt{NuFissionMatrixXS} & \cmark & & \\
  \texttt{ScatterXS} & \cmark & \cmark & \cmark \\
  \texttt{ScatterMatrixXS} & \cmark & & \\
  \texttt{ScatterProbabilityMatrixXS} & \cmark & & \\
  \texttt{TotalXS} & \cmark & \cmark & \cmark \\
  \texttt{TransportXS} & \cmark & & \\
  \midrule
  \multicolumn{4}{c}{\bf Delayed Neutron Constants} \\
  \midrule
  \texttt{Beta} & \cmark & \cmark & \cmark \\
  \texttt{ChiDelayed} & \cmark & & \\
  \texttt{DecayRate} & \cmark & \cmark & \cmark \\
  \texttt{DelayedNuFissionXS} & \cmark & \cmark & \cmark \\
  \texttt{DelayedNuFissionMatrixXS} & \cmark & & \\
  \bottomrule
\end{tabular}
\end{table}

%%%%%%%%%%%%%%%%%%%%%%%%%%%%%%%%%%%%%%%%%%%%%%%%%%%%%%%%%%%%%%%%%%%%%%%%%%%%%%%
\section{Results}
\label{sec:results}
%%%%%%%%%%%%%%%%%%%%%%%%%%%%%%%%%%%%%%%%%%%%%%%%%%%%%%%%%%%%%%%%%%%%%%%%%%%%%%%


%%%%%%%%%%%%%%%%%%%%%%%%%%%%%%%%%%%%%%%%%%%%%%%%%%%%%%%%%%%%%%%%%%%%%%%%%%%%%%%
\subsection{A Single-Step Framework for MGXS Generation}
\label{sec:single-step}
%%%%%%%%%%%%%%%%%%%%%%%%%%%%%%%%%%%%%%%%%%%%%%%%%%%%%%%%%%%%%%%%%%%%%%%%%%%%%%%

In general, MGXS generation schemes use a multi-step approach to decouple the energy, angular and spatial dimensions of the transport equation. The multi-step approach typically applies high-fidelity models of the energy self-shielding physics to low-fidelity geometric models of unique core components as illustrated in \autoref{fig:multi-step-framework}. The multi-step approach uses a combination of models of varying complexity to optimize overall simulation speed with accuracy. However, this is often done at the expense of generality. For example, multi-step MGXS generation schemes do not typically model inter-assembly physics or the effect of reflectors and other core heterogeneities on the spatial distribution of the flux. The approximations to the energy and spatial variation of the flux introduce approximation error in full-core calculations and limit the core design parameter space for which multi-step schemes may be applied. 

\begin{figure}[h!]
\centering
\includegraphics[width=0.8\linewidth]{figures/multi-step-flow-chart}
\caption{The multi-step approach typically used for multi-group reactor physics calculations \cite{gibson2016thesis}.}
\label{fig:multi-step-framework}
\end{figure}

Monte Carlo-based MGXS generation methods to date have retained the multi-step geometric framework to tabulate MGXS for individual reactor components -- such as infinite fuel pins and/or assemblies -- for subsequent use in full-core multi-group calculations. Although the use of MC within a multi-step framework eliminates the need to approximate the flux in energy, it does not account for spatial self-shielding effects throughout a reactor core. One novel aspect of the \texttt{openmc.mgxs} module is that it can be easily configured in a single-step framework that uses OpenMC eigenvalue simulations of the complete heterogeneous geometry to simultaneously account for all energy and spatial effects in a single step. The single-step framework may be impractical for MGXS generation for industrial applications since it is constrained by the slow convergence rate of Monte Carlo tallies. Nevertheless, it allows for the rigorous quantification of approximation error due to energy and spatial self-shielding models used to generate MGXS. Alternatively, the \texttt{openmc.mgxs} module can be also be configured within a traditional multi-step framework to generate MGXS for infinite fuel pins or assemblies. Both single-step and multi-step frameworks are evaluated for two heterogeneous PWR benchmarks in the following sections.


%%%%%%%%%%%%%%%%%%%%%%%%%%%%%%%%%%%%%%%%%%%%%%%%%%%%%%%%%%%%
\subsection{MGXS Generation with OpenMC}
\label{subsec:openmc}

-describe infinite vs. null approaches

OpenMC was employed to generate MGXS, and reference eigenvalues and pin-wise fission and U-238 capture rates. The MGXS were tallied in CASMO's seventy energy group structure \cite{rhodes2006casmo} from a single eigenvalue calculation. The OpenMC simulations were performed with 1000 batches with 10$^{6}$ particle histories per batch. Stationarity of the fission source was obtained with 100 inactive batches for both benchmarks.

The OpenMC simulations used the ``iso-in-lab'' feature to enforce isotropic in lab scattering. The ``iso-in-lab'' feature samples the outgoing neutron energy from the scattering laws prescribed by the continuous energy cross section library, but the outgoing neutron direction of motion is sampled from an isotropic in lab distribution. Although isotropic in lab scattering is a poor approximation for LWRs, it eliminated scattering source anisotropy as one possible cause of approximation error between OpenMC and OpenMOC. 


%%%%%%%%%%%%%%%%%%%%%%%%%%%%%%%%%%%%%%%%%%%%%%%%%%%%%%%%%%%%
\subsection{Multi-Group Calculations with OpenMOC}
\label{subsec:openmoc}

The OpenMOC code \cite{boyd2014openmoc} was employed to use the MGXS generated by OpenMC for deterministic multi-group transport calculations. The OpenMOC code is a 2D method of characteristics code designed for fixed source and eigenvalue neutron transport calculations. OpenMOC approximates the scattering source as isotropic in the lab coordinate system, and discretizes the geometry into flat source regions (FSRs) which approximate the neutron source as constant across each spatial zone. The OpenMOC eigenvalue and energy-integrated, pin-wise reaction rates were compared with the reference solutions computed by OpenMC. Each OpenMOC simulation used a characteristic track laydown with 128 azimuthal angles and 0.05 cm spacing, and was converged to 10$^{-5}$ on the root mean square of the energy-integrated fission source in each FSR. Coarse Mesh Finite Difference acceleration was applied on a pin-wise spatial mesh to reduce the number of iterations required to converge the fine-mesh transport calculations.


%%%%%%%%%%%%%%%%%%%%%%%%%%%%%%%%%%%%%%%%%%%%%
\subsection{Benchmarks and Reference Results}
\label{subsec:benchmarks}

This paper modeled benchmarks derived from the Benchmark for Evaluation And Validation of Reactor Simulations (BEAVRS) PWR model~\cite{horelik2013beavrs}. Each test case includes heterogeneous features -- and corresponding spatial self-shielding effects -- to demonstrate the potential utility of a single-step framework for MGXS generation. The benchmarks were comprised of materials from the BEAVRS model, including 1.6\% and 3.1\% enriched UO$_2$ fuel, borated water (975 ppm boron), zircaloy, helium, air, borosilicate glass and stainless steel. Each material was modeled with cross sections from the ENDF/B-VII.1 continuous energy cross section library~\cite{mcnpx2003manual} evaluated at 600K for hot zero power conditions. 

%Although BEAVRS is an axially heterogeneous 3D core model, both benchmarks were fabricated in 2D due to the geometric constraints in OpenMOC.

The first benchmark was a single fuel assembly with an array of 264 fuel pins of 1.6\% enriched UO$_2$ fuel with zircaloy cladding and a helium gap. The assembly included 24 control rod guide tubes (CRGTs) filled by borated water and surrounded by zircaloy cladding, and a central instrument tube filled with air surrounded by two zircaloy tubes separated by borated water. The second benchmark was constructed as a 2$\times$2 colorset of two fuel assemblies extracted from the BEAVRS model. The top-left and bottom-right fuel assemblies were of the same enrichment and configuration as the single assembly benchmark. The top-right and bottom-left fuel assemblies included 264 fuel pins of 3.1\% enriched UO$_2$ fuel, 20 CRGTs and a central instrument tube. In addition, the two 3.1\% enriched assemblies included four burnable poisons (BPs) consisting of eight layers of air, steel, borosilicate glass and zircaloy. The colorset was surrounded by a water reflector on the bottom and right that was of the same width as a fuel assembly. The assembly benchmark was modeled with reflective boundary conditions, while the colorset was modeled with reflective boundaries on the top and left and vacuum boundaries on the bottom and right. The assembly and colorset are illustrated in \autoref{fig:benchmarks-materials}.

%The intra-pin grid spacers and grid sleeves separating each assembly in the BEAVRS model were not included either benchmark. 

\begin{figure}[h!]
\centering
\begin{subfigure}{0.42\textwidth}
  \centering
  \includegraphics[width=0.8\linewidth]{figures/assembly/geometry}
  \caption{}
  \label{fig:benchmarks}
\end{subfigure}
\begin{subfigure}{0.42\textwidth}
  \centering
  \includegraphics[width=0.8\linewidth]{figures/colorset/geometry}
  \caption{}
  \label{fig:benchmarks-colorset}
\end{subfigure}
\caption{The assembly (a) and colorset (b) benchmark geometries.}
\label{fig:benchmarks-materials}
\end{figure}

\begin{figure}[h!]
\centering
\begin{subfigure}{0.42\textwidth}
  \centering
  \includegraphics[width=0.8\linewidth]{figures/assembly/fsrs}
  \caption{}
  \label{fig:benchmarks-assm-fsrs}
\end{subfigure}%
\begin{subfigure}{0.42\textwidth}
  \centering
  \includegraphics[width=0.8\linewidth]{figures/colorset/fsrs}
  \caption{}
  \label{fig:benchmarks-colorset-fsrs}
\end{subfigure}
\caption{OpenMOC flat source regions for the assembly (a) and colorset (b) benchmarks.}
\label{fig:benchmarks-fsrs}
\end{figure}

Flat source region spatial discretization meshes were applied to both benchmarks for the OpenMOC simulations as shown in \autoref{fig:benchmarks-fsrs}. The UO$_2$ fuel was subdivided into five equal volume radial rings, while ten radial rings were employed in the water-filled CRGTs and instrument tubes. The borosilicate glass and borated water material zones filling the BPs were each discretized into five equal volume radial rings. Five equally spaced rings were used in the moderator zones surrounding each pin. Eight equal angle subdivisions were used in all pin cell material zones. The 13.85824 cm of water reflector nearest the fuel assemblies in the colorset benchmark was discretized in a 0.125984 cm $\times$ 0.125984 cm rectilinear mesh, equivalent to a 10$\times$10 mesh in each pin. The 7.55904 cm of reflector furthest from the fuel assemblies was discretized in a 1.25984 cm $\times$ 1.25984 cm pin-wise mesh.

A series of OpenMC simulations was used to calculate reference eigenvalues, pin-wise fission rates, and pin-wise U-238 capture rates for both benchmarks. The reference solutions were computed with 100 inactive and 900 active batches of 10$^7$ particle histories per batch. The OpenMC ``combined'' eigenvalue estimator is reported along with the associated 1-sigma uncertainty of one pcm for both benchmarks in \autoref{tab:keff-reference}. The reference energy-integrated fission and U-238 capture rate spatial distributions were computed using rectilinear, pin-wise tally meshes in OpenMC and are shown in \autoref{fig:benchmarks-rxn-rates}. The reaction rates were normalized to the mean of all non-zero reaction rates. The rates in the instrument tubes, CRGTs and BPs are all zero and are shaded in white. The 1-sigma uncertainties are less than 0.08\% in each pin for both benchmarks.

\begin{table}[h!]
  \centering
  \caption{Reference OpenMC eigenvalues for each benchmark.}
  \label{tab:keff-reference} 
  \begin{tabular}{c c}
  \toprule
  {\bf Assembly} &
  {\bf Colorset} \\
  \midrule
  0.99326 $\pm$ 0.00001 & 0.94574 $\pm$ 0.00001 \\
  \bottomrule
\end{tabular}
\end{table}

\begin{figure*}[h!]
\centering
\begin{subfigure}{0.45\textwidth}
  \includegraphics[width=\linewidth]{figures/assembly/fission-rates}
  \caption{}
  \label{fig:fiss-assm}
\end{subfigure}%
\begin{subfigure}{0.45\textwidth}
  \includegraphics[width=\linewidth]{figures/assembly/capture-rates}
  \caption{}
  \label{fig:capt-assm}
\end{subfigure}
\begin{subfigure}{0.45\textwidth}
  \centering
  \includegraphics[width=\linewidth]{figures/colorset/fission-rates}
  \caption{}
  \label{fig:fiss-colorset}
\end{subfigure}%
\begin{subfigure}{0.45\textwidth}
  \centering
  \includegraphics[width=\linewidth]{figures/colorset/capture-rates}
  \caption{}
  \label{fig:capt-colorset}
\end{subfigure}
\caption{Reference OpenMC fission and U-238 capture rates for the assembly (a) -- (b) and colorset (c) -- (d) benchmarks.}
\label{fig:benchmarks-rxn-rates}
\end{figure*}


%%%%%%%%%%%%%%%%%%%%%%%%%%%%%%%%%%%%%%%%%%%%%%%%%%%%%%%%%%%%%%%%%%%%%%%%%%%%%%%
\subsection{Results}
\label{subsec:results}
%%%%%%%%%%%%%%%%%%%%%%%%%%%%%%%%%%%%%%%%%%%%%%%%%%%%%%%%%%%%%%%%%%%%%%%%%%%%%%%

Both benchmarks were modeled with OpenMOC using MGXS generated by OpenMC using the single-step framework. The OpenMOC eigenvalues were compared to the reference OpenMC eigenvalues from \autoref{tab:keff-reference}. The eigenvalue bias $\Delta\rho$ was calculated by comparing the eigenvalue $k_{eff}^{MOC}$ from OpenMOC to the reference eigenvalue $k_{eff}^{MC}$ computed by OpenMC in units of per cent mille (pcm):

\begin{equation}
\label{eqn:delta-rho}
\Delta\rho = \left(k_{eff}^{MOC} - k_{eff}^{MC}\right) \times 10^{5}
\end{equation}

The bias is listed for both benchmarks in \autoref{tab:keff-bias}. The slightly negative bias of a few hundred pcm is likely due to the flux separability approximation \cite{boyd2017sph}, which permits use of the scalar rather than the angular neutron flux to collapse cross sections. 

%It should be recalled that isotropic in lab scattering is used by OpenMC to compute both the reference solution and the MGXS. If anisotropic scattering were employed in OpenMC, one would expect quite different biases without a robust implementation of a higher order scattering kernel in OpenMOC.

\begin{table}[h!]
  \centering
  \caption{OpenMOC eigenvalue bias $\Delta\rho$.}
  \label{tab:keff-bias} 
  \begin{tabular}{l l r r r}
  \toprule
  \textbf{Benchmark} & \textbf{MGXS Scheme} & \textbf{2-Group} & \textbf{8-Group} & \textbf{70-Group} \\
  \midrule
  \multirow{2}{*}{Assembly} & Infinite    & -132 & -68 &   31 \\
                            & Single-Step &   60 & -72 & -161 \\
  \midrule
  \multirow{2}{*}{Colorset} & Infinite    & 2103 & 267 &   46 \\
                            & Single-Step & 1818 & 478 & -142 \\
  \bottomrule
\end{tabular}
\end{table}

The OpenMOC energy-integrated pin-wise fission rates were compared to the reference OpenMC fission rates \autoref{fig:benchmarks-rxn-rates}. The percent relative errors for each pin's fission rates were computed and the maximum and mean errors are listed in \autoref{tab:fiss-max-errors} and \autoref{tab:fiss-mean-errors}, respectively. In particular, the maximum errors are the maximum of the absolute values of the errors along with the appropriate sign, while the mean errors are the averages of the absolute error magnitudes.

\begin{table}[h!]
  \centering
  \caption{OpenMOC maximum fission rate percent relative errors.}
  \label{tab:fiss-max-errors}
  \begin{tabular}{l l c c c}
  \toprule
  \textbf{Benchmark} & \textbf{MGXS Scheme} & \textbf{2-Group} & \textbf{8-Group} & \textbf{70-Group} \\
  \midrule
  \multirow{2}{*}{Assembly} & Infinite    & 2.387 & 0.643 & 0.375 \\
                            & Single-Step & 2.379 & 0.638 & 0.380 \\
  \midrule
  \multirow{2}{*}{Colorset} & Infinite    &  11.024 &  2.773 & 0.670 \\
                            & Single-Step & -16.330 & -2.855 & 0.764 \\
  \bottomrule
\end{tabular}
\end{table}

\begin{table}[h!]
  \centering
  \caption{OpenMOC mean absolute fission rate percent relative errors.}
  \label{tab:fiss-mean-errors}
  \begin{tabular}{l l c c c}
  \toprule
  \textbf{Benchmark} & \textbf{MGXS Scheme} & \textbf{2-Group} & \textbf{8-Group} & \textbf{70-Group} \\
  \midrule
  \multirow{2}{*}{Assembly} & Infinite    & 0.951 & 0.231 & 0.073 \\
                            & Single-Step & 0.943 & 0.229 & 0.074 \\
  \midrule
  \multirow{2}{*}{Colorset} & Infinite    & 4.964 & 1.029 & 0.147 \\
                            & Single-Step & 5.471 & 1.080 & 0.178 \\
  \bottomrule
\end{tabular}
\end{table}

The spatial distributions of fission rate errors for the single-step framework are plotted as heatmaps for each benchmark in \autoref{fig:fiss-errors}. The heatmaps illustrate systematic trends in the pin-wise fission errors which correlate with spatial heterogeneities in each benchmark. In particular, the fission rates are generally underpredicted for pins adjacent to a single CRGT, but overpredicted for pins adjacent to two CRGTs in the assembly. In addition, the errors are largest for pins along the inter-assembly and assembly-reflector interfaces for the colorset benchmark.

%For the PWR benchmarks modeled here, the moderation provided by neighboring CRGTs and reflectors softens the flux for nearby fuel pins and should be modeled when collapsing pin-wise MGXS for high-fidelity multi-group transport calculations.

\begin{figure*}[h!]
\centering
\begin{subfigure}{0.45\textwidth}
  \centering
  \includegraphics[width=\linewidth]{figures/assembly/fiss-single-step-errors}
  \caption{}
  \label{fig:assm-fiss-single-step-error}
\end{subfigure}%
\begin{subfigure}{0.45\textwidth}
  \centering
  \includegraphics[width=\linewidth]{figures/colorset/fiss-single-step-errors}
  \caption{}
  \label{fig:colorset-fiss-single-step-error}
\end{subfigure}
\caption{OpenMOC fission rate percent relative errors for the assembly (a) and colorset (b) benchmarks.}
\label{fig:fiss-errors}
\end{figure*}

The OpenMOC energy-integrated pin-wise U-238 capture rates were compared to the reference OpenMC capture rates shown in \autoref{fig:benchmarks-rxn-rates}. The percent relative errors for each pin's capture rates were computed and the maximum and mean errors are listed in \autoref{tab:capt-max-errors} and \autoref{tab:capt-mean-errors}, respectively. In particular, the maximum errors are the maximum of the absolute values of the errors along with the appropriate sign, while the mean errors are the averages of the absolute error magnitudes.

\begin{table}[h!]
  \centering
  \caption{OpenMOC maximum U-238 capture rate percent relative errors.}
  \label{tab:capt-max-errors}
  \begin{tabular}{l l c c c}
  \toprule
  \textbf{Benchmark} & \textbf{MGXS Scheme} & \textbf{2-Group} & \textbf{8-Group} & \textbf{70-Group} \\
  \midrule
  \multirow{2}{*}{Assembly} & Infinite    & -2.644 & -1.480 & -1.102 \\
                            & Single-Step & -2.629 & -1.475 & -1.101 \\
  \midrule
  \multirow{2}{*}{Colorset} & Infinite    & 12.010 & 3.618 & -1.889 \\
                            & Single-Step & 11.100 & 3.372 & -1.969 \\
  \bottomrule
\end{tabular}
\end{table}

\begin{table}[h!]
  \centering
  \caption{OpenMOC mean absolute U-238 capture rate percent relative errors.}
  \label{tab:capt-mean-errors}
  \begin{tabular}{l l c c c}
  \toprule
  \textbf{Benchmark} & \textbf{MGXS Scheme} & \textbf{2-Group} & \textbf{8-Group} & \textbf{70-Group} \\
  \midrule
  \multirow{2}{*}{Assembly} & Infinite    & 1.252 & 0.643 & 0.480 \\
                            & Single-Step & 1.247 & 0.641 & 0.479 \\
  \midrule
  \multirow{2}{*}{Colorset} & Infinite    & 3.878 & 0.847 & 0.480 \\
                            & Single-Step & 3.708 & 0.780 & 0.478 \\
  \bottomrule
\end{tabular}
\end{table}

The spatial distributions of capture rate errors are plotted as heatmaps for each benchmark in \autoref{fig:capt-errors}. The heatmaps illustrate systematic error trends in the pin-wise capture errors which correlate with spatial heterogeneities in each benchmark. This underscores the importance of accounting for spatial heterogeneities -- such as the added moderation from CRGTs and reflectors -- when generating MGXS to predict U-238 capture and Pu-239 production in LWRs. The moderation provided by neighboring CRGTs and/or reflectors softens the local flux for nearby fuel pins and should be modeled when collapsing pin-wise MGXS for high-fidelity multi-group transport calculations. 

\begin{figure*}[h!]
\centering
\begin{subfigure}{0.45\textwidth}
  \centering
  \includegraphics[width=\linewidth]{figures/assembly/capt-single-step-errors}
  \caption{}
  \label{fig:assm-capt-single-step-error}
\end{subfigure}%
\begin{subfigure}{0.45\textwidth}
  \centering
  \includegraphics[width=\linewidth]{figures/colorset/capt-single-step-errors}
  \caption{}
  \label{fig:colorset-capt-single-step-error}
\end{subfigure}
\caption{OpenMOC U-238 capture rate percent relative errors for the assembly (a) and colorset (b) benchmarks.}
\label{fig:capt-errors}
\end{figure*}

%%%%%%%%%%%%%%%%%%%%%%%%%%%%%%%%%%%%%%%%%%%%%%%%%%%%%%%%%%%%%%%%%%%%%%%%%%%%%%%
\section{Conclusions}
\label{sec:conclusions}
%%%%%%%%%%%%%%%%%%%%%%%%%%%%%%%%%%%%%%%%%%%%%%%%%%%%%%%%%%%%%%%%%%%%%%%%%%%%%%%

Monte Carlo methods are increasingly used to generate multigroup cross sections for coarse mesh neutron diffusion codes. This paper introduced the \texttt{openmc.mgxs} Python module to generate MGXS with the OpenMC Monte Carlo code for neutron transport applications. This new module utilizes OpenMC's tally system to perform stochastic integration of reaction rates and fluxes for a user-specified energy group discretization and spatial mesh. The different types of MGXS computable to date---including standard group-wise constants, as well as prompt and delayed constants---were tabulated here. The \texttt{openmc.mgxs} module leverages OpenMC's Python API, along with its support for tally slicing, merging, and arithmetic, to provide a scalable data-processing framework complementary to but separate from the transport kernel implemented in the OpenMC executable.

A case study used the multigroup OpenMOC transport code with 70-group MGXS generated by OpenMC to model two heterogeneous PWR benchmarks. The study showed that OpenMOC predicted eigenvalues to within 50 pcm and fission rates to within 1\% of reference solutions computed by OpenMC, demonstrating the efficacy of the \texttt{openmc.mgxs} module to enable highly accurate multigroup transport calculations. \textcolor{red}{The multi-group constants generated by OpenMC are intended for use by fine-mesh transport rather than coarse-mesh diffusion codes, which brings its own set of unique -- though not mutually exclusive -- challenges with respect to few-group diffusion.} We expect that \texttt{openmc.mgxs} may prove to be a useful platform for research to address these challenges for MC-based MGXS generation in the future.

%A fine spatial and/or energy discretization may be used by OpenMC to tabulate MGXS, and subsequently condensed in energy and/or homogenized in space with data processing features in Python.

%%%%%%%%%%%%%%%%%%%%%%%%%%%%%%%%%%%%%%%%%%%%%%%%%%%%%%%%%%%%%%%%%%%%%%%%%%%%%%%
\section*{Acknowledgments}
%%%%%%%%%%%%%%%%%%%%%%%%%%%%%%%%%%%%%%%%%%%%%%%%%%%%%%%%%%%%%%%%%%%%%%%%%%%%%%%

The first author was supported by the Idaho National Laboratory and the National Science Foundation Graduate Research Fellowship Grant No. 1122374.

%\section*{References}
\setlength{\baselineskip}{12pt}
\bibliography{references}
\bibliographystyle{ans}

\end{document}
