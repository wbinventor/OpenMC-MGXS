%%%%%%%%%%%%%%%%%%%%%%%%%%%%%%%%%%%%%%%%%%%%%%%%%%%%%%%%%%%%%%%%%%%%%%%%%%%%%%%
\section{MGXS Module Design}
\label{sec:design}
%%%%%%%%%%%%%%%%%%%%%%%%%%%%%%%%%%%%%%%%%%%%%%%%%%%%%%%%%%%%%%%%%%%%%%%%%%%%%%%

The OpenMC Python API features a module called \texttt{openmc.mgxs} that was designed to generate multigroup cross sections. The \texttt{openmc.mgxs} module is built atop the underlying core features in the rest of the API to support a seamless interface for both input generation and downstream data processing of MGXS from Python. In particular, one may specify the MGXS to compute and the \texttt{openmc.mgxs} module will construct the necessary \texttt{Tally} objects. The \texttt{Tally} objects may be easily exported to XML input files and used to containerize and process the tally data produced by an OpenMC simulation. The \texttt{openmc.mgxs} module thereby leverages the existing classes and features (\textrm{e.g.}, tally arithmetic and Pandas DataFrames) provided by the Python API.

The \texttt{openmc.mgxs} module uses an object-oriented design based on an abstract \texttt{MGXS} class with subclasses for different cross section types. The \texttt{MGXS} subclasses, as listed in \cref{tab:mgxs-types}, compute macroscopic or microscopic multigroup constants in one or more arbitrary energy group structures from MC tallies. The \texttt{openmc.mgxs} module also includes a \texttt{Library} class that automates the construction of \texttt{MGXS} objects for different group structures, spatial domains, and reaction types.

\begin{table}[h!]
  \centering
  \caption{Multigroup cross section types implemented by the \texttt{openmc.mgxs} module.}
  \small
  \label{tab:mgxs-types}
  \vspace{6pt}
  \begin{tabular}{l l}
  \toprule
  \textbf{Class} &
  \textbf{Description} \\
  \midrule
  \multicolumn{2}{c}{\bf Prompt Neutron Constants} \\
  \midrule
  \texttt{AbsorptionXS} & Absorption \\
  \texttt{Chi} & Fission emission spectrum \\
  \texttt{FissionXS} & Fission \\
  \texttt{InverseVelocity} & Inverse neutron velocity \\
  \texttt{KappaFissionXS} & Fission energy release \\
  \texttt{MultiplicityMatrixXS} & Scattering multiplicity matrix \\
  \texttt{NuFissionMatrixXS} & Fission production matrix \\
  \texttt{ScatterXS} & Scattering \\
  \texttt{ScatterMatrixXS} & Scattering matrix \\
  \texttt{ScatterProbabilityMatrixXS} & Scattering probability matrix \\
  \texttt{TotalXS} & Total collision \\
  \texttt{TransportXS} & Transport-corrected total collision \\
  \midrule
  \multicolumn{2}{c}{\bf Delayed Neutron Constants} \\
  \midrule
  \texttt{Beta} & Delayed neutron fraction \\
  \texttt{ChiDelayed} & Delayed fission spectrum \\
  \texttt{DecayConstant} & Delayed neutron precursor decay constant \\
  \texttt{DelayedNuFissionXS} & Fission delayed neutron production \\
  \texttt{DelayedNuFissionMatrixXS} & Fission delayed neutron production matrix \\
  \bottomrule
\end{tabular}
\end{table}


%%%%%%%%%%%%%%%%%%%%%%%%%%%%%
\subsection{Workflow}
\label{subsec:workflow}

The MGXS generation workflow begins by creating instances of \texttt{MGXS} subclasses. In general, it is assumed that the model geometry and materials are assumed to have already been defined. Two options exist for instantiating \texttt{MGXS} objects: (1) manual instantiation of the subclasses or (2) automated instantiation via the \texttt{Library} class. In the first option, an instance of an \texttt{MGXS} subclass is assigned a spatial domain and an energy-group structure. One can also specify whether cross sections are desired for all nuclides or on a per nuclide basis through the \texttt{MGXS.by_nuclide} attribute. Tallies are produced automatically via a \texttt{MGXS.tallies} attribute and can be appended to a collection of tallies represented by a \texttt{Tallies} class object. The tallies collection in turn has an \texttt{export_to_xml()} method that writes out the XML input file to be read by the transport solver. Alternatively, the \texttt{Library} class allows one to specify that multiple cross sections should be computed for multiple spatial domains and a given energy-group structure. The \texttt{Library.add_to_tallies_file()} method then adds all necessary tallies to compute the MGXS to an existing tallies collection, merging together any tallies that can be merged to reduce the total number of tallies.

After a simulation has completed, an HDF5 \emph{state point} file is written
that contains the tally results. From the Python API, the \texttt{StatePoint}
object reads in the tally results and can then be used by the
\texttt{MGXS.load_from_statepoint()} method, which loads those results into an
\texttt{MGXS} object. At that point, the multigroup cross sections can be displayed on standard output, saved to a file, or converted to a Pandas DataFrame, as described further below. Two examples of the workflow for generating MGXS manually and using the \texttt{Library} are shown in \cref{lst:mgxs-manual} and \cref{lst:mgxs-library}, respectively.

\lstinputlisting[language=Python, basicstyle=\ttfamily\scriptsize, caption={MGXS calculation with manual object instantiation.}, label={lst:mgxs-manual}]{listings/mgxs-manual.py}

\lstinputlisting[language=Python, basicstyle=\ttfamily\scriptsize, caption={MGXS calculation with \texttt{Library}-automated object instantiation.}, label={lst:mgxs-library}]{listings/mgxs-library.py}


%%%%%%%%%%%%%%%%%%%%%%%%%%%%%
\subsection{Data Processing and Storage}
\label{subsec:data-processing}

The \texttt{openmc.mgxs} module takes advantage of many of the existing features of the Python API discussed in \cref{sec:openmc}. When a user loads tally results from a statepoint and displays them or stores them to disk, the following sequence of events happens.

\begin{itemize}[noitemsep]
\item Reaction rate and flux \texttt{Tally} objects are obtained by using tally slicing. This step is necessary because during input generation, the tallies for multiple \texttt{MGXS} objects may have been merged into fewer tallies. Slicing returns a \texttt{Tally} object with only the filters, nuclides, and scores required for the desired MGXS.
\item A derived cross section \texttt{Tally} object is produced by using tally arithmetic to divide the reaction rate tally by the flux tally for all energy groups, nuclides, and spatial homogenization zones. The time to perform these operations is minimal since the tally results are stored in dense NumPy arrays that support vector arithmetic.
\item If a user requests a Pandas DataFrame for an \texttt{MGXS} object, it is produced by first generating a Pandas DataFrame for the derived cross section tally and then modifying it with a few user-friendly customizations for MGXS applications (\textrm{e.g.}, labeling energy groups).
\end{itemize}

The \texttt{openmc.mgxs} module was developed with general design principles to generate MGXS for any multigroup neutron transport code. Although the module does not explicitly support any multigroup codes, it can export MGXS data to a variety of data storage formats, including omma-separated values (CSV) and HDF5. The exported MGXS files may be easily transformed into the database or input files required by a particular multigroup code. Additionally, a cross section library can be exported to the format used by OpenMC's multigroup transport mode.

%%%%%%%%%%%%%%%%%%%%%%%%%%%%%%%%%%%%%%%%%%%%%%%%%%%%%%%%%%%%%%%%%%%%%%%%%%%%%%%
\subsection{Additional Features}
\label{sec:features}
%%%%%%%%%%%%%%%%%%%%%%%%%%%%%%%%%%%%%%%%%%%%%%%%%%%%%%%%%%%%%%%%%%%%%%%%%%%%%%%


%%%%%%%%%%%%%%%%%%%%%%%%%%%%%%%%%%%%%%%%%%%%
\subsubsection{Energy Condensation}
\label{subsec:energy-condense}

The \texttt{openmc.mgxs} module supports energy condensation in downstream data processing, which is useful for exploring approximation bias in various energy group structures. For example, MGXS may be computed in some ``fine'' energy group structure and the tally data subsequently condensed to some coarser-group structure \texttt{coarse_groups} for multigroup calculations with the \texttt{MGXS.get_condensed_xs(coarse_groups)} Python method. Energy condensation may be performed to arbitrarily defined coarse-group structures provided the coarse-group boundaries coincide with boundaries in the fine-group structure.


%%%%%%%%%%%%%%%%%%%%%%%%%%%%%%%%%%%%%%%%%%%%
\subsubsection{Spatial Homogenization}
\label{subsec:pinwise-homogenize}

The \texttt{openmc.mgxs} module is designed to perform spatial homogenization on heterogeneous tally meshes for fine-mesh transport codes. In OpenMC parlance, MGXS may be computed for material, cell, or universe spatial domains. In addition, the module supports MGXS calculations for repeated cell instances using distribcell spatial tally domains\cite{lax2014distribcell}. Spatial homogenization across some subset of distributed cell instances \texttt{cell_instances} can be performed by using the \texttt{MGXS.get_subdomain_avg_xs(cell_instances)} Python class method.

%The \texttt{openmc.mgxs} module may also perform spatial homogenization on structured Cartesian tally meshes for coarse-mesh multigroup calculations..


%%%%%%%%%%%%%%%%%%%%%%%%%%%%%%%%%%%%%%%%%%%%
\subsubsection{Microscopic MGXS}
\label{subsec:micro-macro}

When tally results are loaded into an \texttt{MGXS} object, the reaction rate tallies contain macroscopic quantities. Thus, by default, when a derived cross section tally is computed by dividing reaction rates by fluxes, a macroscopic cross section is obtained. However, all methods that output MGXS data (\textrm{e.g.}, \texttt{MGXS.print_xs()}, \texttt{MGXS.get_pandas_dataframe()}, \texttt{MGXS.export_xs_data()}) also have an argument that allows the user to specify that microscopic cross sections should be calculated. This option works both for total material cross sections and for per nuclide cross sections.

Microscopic cross sections are computed for spatial domains with homogeneous material composition by dividing by the known isotopic number densities. Furthermore, microscopic cross sections can be calculated for spatial domains with heterogeneous material composition by using OpenMC's stochastic volume calculation mode. The stochastic volume calculation defines a bounding box for a spatial domain of interest, samples points uniformly within the bounding box, and counts how many points lie within the spatial domain, thereby estimating the volume fraction of the domain within the bounding box. Such a capability can be useful when microscopic MGXS are desired over an entire assembly or other spatial domain of similar complexity.

%%%%%%%%%%%%%%%%%%%%%%%%%%%%%%%%%%%%%%%%
\subsubsection{Isotropic in Lab Scattering}
\label{subsec:iso-in-lab}

A unique option for treating all scattering as isotropic in the laboratory frame of reference is implemented in the OpenMC code. This feature, abbreviated as \emph{iso-in-lab}, may be useful to quantify the ability of multigroup codes in order to capture anisotropic scattering effects with higher-order scattering matrices or transport correction schemes. The iso-in-lab feature is implemented as an optional attribute for each nuclide or element in a simulation. When iso-in-lab scattering is specified for a nuclide or element, the outgoing neutron energy is sampled from the scattering laws prescribed by the continuous-energy cross section library, but the outgoing neutron direction of motion is sampled from a distribution that is isotropic in the laboratory frame.

%%%%%%%%%%%%%%%%%%%%%%%%%%%%%
\subsection{Tally Estimators}
\label{subsec:tally-est}

The \texttt{openmc.mgxs} module supports a variety of tally estimators for each MGXS type as shown in \cref{tab:mgxs-tally-estimators}. Tallies with \textit{analog} estimators are incremented only when the scoring function of interest (\textrm{i.e.}, reaction type) is explicitly sampled. In contrast, tallies with \textit{collision} and \textit{track-length} estimators are incremented for all scoring functions every time the appropriate region of phase space is sampled (\textrm{e.g.}, energy group) in order to improve the sample statistics. All MGXS types currently support analog estimators, and most support collision and track-length estimators. However, all MGXS types that require a tally over outgoing energy (\textrm{e.g.}, scattering matrix, fission spectrum) currently support only analog estimators. Future development of the \texttt{openmc.mgxs} module may implement a methodology\cite{nelson2014improved} that permits track-length estimators for tallies that depend on the outgoing neutron energy.

\begin{table}[h!]
  \centering
  \caption{Tally estimators supported by each MGXS type.}
  \small
  \label{tab:mgxs-tally-estimators}
  \vspace{6pt}
  \begin{tabular}{l c c c}
  \toprule
  \textbf{Class} &
  \textbf{Analog} &
  \textbf{Collision} &
  \textbf{Track-length} \\
  \midrule
  \multicolumn{4}{c}{\bf Prompt Neutron Constants} \\
  \midrule
  \texttt{AbsorptionXS} & \cmark & \cmark & \cmark \\
  \texttt{CaptureXS} & \cmark & \cmark & \cmark \\
  \texttt{Chi} & \cmark & & \\
  \texttt{FissionXS} & \cmark & \cmark & \cmark \\
  \texttt{InverseVelocity} & \cmark & \cmark & \cmark \\
  \texttt{KappaFissionXS} & \cmark & \cmark & \cmark \\
  \texttt{MultiplicityMatrixXS} & \cmark & & \\
  \texttt{NuFissionMatrixXS} & \cmark & & \\
  \texttt{ScatterXS} & \cmark & \cmark & \cmark \\
  \texttt{ScatterMatrixXS} & \cmark & & \\
  \texttt{ScatterProbabilityMatrixXS} & \cmark & & \\
  \texttt{TotalXS} & \cmark & \cmark & \cmark \\
  \texttt{TransportXS} & \cmark & & \\
  \midrule
  \multicolumn{4}{c}{\bf Delayed Neutron Constants} \\
  \midrule
  \texttt{Beta} & \cmark & \cmark & \cmark \\
  \texttt{ChiDelayed} & \cmark & & \\
  \texttt{DecayConstant} & \cmark & \cmark & \cmark \\
  \texttt{DelayedNuFissionXS} & \cmark & \cmark & \cmark \\
  \texttt{DelayedNuFissionMatrixXS} & \cmark & & \\
  \bottomrule
\end{tabular}
\end{table}
