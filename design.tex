%%%%%%%%%%%%%%%%%%%%%%%%%%%%%%%%%%%%%%%%%%%%%%%%%%%%%%%%%%%%%%%%%%%%%%%%%%%%%%%
\section{MGXS Module Design}
\label{sec:design}
%%%%%%%%%%%%%%%%%%%%%%%%%%%%%%%%%%%%%%%%%%%%%%%%%%%%%%%%%%%%%%%%%%%%%%%%%%%%%%%

The Python API features a module called \texttt{openmc.mgxs} that was designed to generate multi-group cross sections. The \texttt{openmc.mgxs} module is built atop the underlying core features in the rest of the API to support a seamless interface for both input generation and downstream data processing of MGXS from Python. In particular, one may specify the MGXS to compute and the \texttt{openmc.mgxs} module will construct the necessary \texttt{Tally} objects. The \texttt{Tally} objects may be easily exported to XML input files and used to containerize and process the tally data produced by an OpenMC simulation. The \texttt{openmc.mgxs} module thereby leverages the existing classes and features (e.g., tally arithmetic and Pandas DataFrames) provided by the Python API.

The \texttt{openmc.mgxs} module uses an object-oriented design based on an abstract \texttt{MGXS} class with subclasses for different cross section types. The \texttt{MGXS} subclasses, as listed in \cref{tab:mgxs-types}, compute macroscopic or microscopic multi-group constants in one or more arbitrary energy group structures from MC tallies. The \texttt{openmc.mgxs} module also includes a \texttt{Library} class which automates the construction of \texttt{MGXS} objects for different group structures, spatial domains, and reaction types.

\begin{table}[h!]
  \centering
  \caption{The multi-group cross section types implemented by the \texttt{openmc.mgxs} module.}
  \small
  \label{tab:mgxs-types}
  \vspace{6pt}
  \begin{tabular}{l l}
  \toprule
  \textbf{Class} &
  \textbf{Description} \\
  \midrule
  \multicolumn{2}{c}{\bf Prompt Neutron Constants} \\
  \midrule
  \texttt{AbsorptionXS} & Absorption \\
  \texttt{CaptureXS} & Radiative capture \\
  \texttt{Chi} & Fission emission spectrum \\
  \texttt{FissionXS} & Fission \\
  \texttt{InverseVelocity} & Inverse neutron velocity \\
  \texttt{KappaFissionXS} & Fission energy release \\
  \texttt{MultiplicityMatrixXS} & Scattering multiplicity matrix \\
  \texttt{NuFissionMatrixXS} & Fission production matrix \\
  \texttt{ScatterXS} & Scattering \\
  \texttt{ScatterMatrixXS} & Scattering matrix \\
  \texttt{ScatterProbabilityMatrixXS} & Scattering probability matrix \\
  \texttt{TotalXS} & Total collision \\
  \texttt{TransportXS} & Transport-corrected total collision \\
  \midrule
  \multicolumn{2}{c}{\bf Delayed Neutron Constants} \\
  \midrule
  \texttt{Beta} & Delayed neutron fraction \\
  \texttt{ChiDelayed} & Delayed fission spectrum \\
  \texttt{DecayRate} & Delayed neutron precursors decay rate \\
  \texttt{DelayedNuFissionXS} & Fission delayed neutron production \\
  \texttt{DelayedNuFissionMatrixXS} & Fission delayed neutron production matrix \\
  \bottomrule
\end{tabular}
\end{table}


%%%%%%%%%%%%%%%%%%%%%%%%%%%%%
\subsection{Workflow}
\label{subsec:workflow}

The MGXS generation workflow begins by creating instances of \texttt{MGXS} subclasses. In general, it is assumed that the model geometry and materials have already been defined. There are two options for creating \texttt{MGXS} instances: 1) by directly using the subclasses, or 2) automated generation via the \texttt{Library} class. In the first option, an instance of a \texttt{MGXS} subclass is assigned a spatial domain and an energy-group structure. One can also specify whether cross sections are desired for the all nuclides or on a per-nuclide basis through the \texttt{MGXS.by_nuclide} attribute. Tallies are produced automatically via a \texttt{MGXS.tallies} attribute and can be appended to an existing collection of tallies. The tallies collection in turn has an \texttt{export_to_xml()} method which writes out the XML input file to be read by the transport solver. Alternatively, the \texttt{Library} class allows one to specify that multiple cross sections should be computed for multiple spatial domains and a given energy-group structure. The \texttt{Library.add_to_tallies_file()} method then adds all necessary tallies to compute the MGXS to an existing tallies collection, merging together any tallies that can be merged to reduce the total number of tallies.

After a simulation has completed, an HDF5 \emph{state point} file is written
that contains the tally results. From the Python API, the \texttt{StatePoint}
object reads in the tally results and can then be used by the
\texttt{MGXS.load_from_statepoint()} method which loads those results into a
\texttt{MGXS} instance. At that point, the multi-group cross sections can be displayed on standard output, saved to file, or converted to a Pandas DataFrame, as described further below. An example of the entire workflow for generating MGXS manually and using the \texttt{Library} is show in \cref{lst:mgxs-manual} and \cref{lst:mgxs-library}, respectively.

\lstinputlisting[language=Python, basicstyle=\ttfamily\scriptsize, caption={Manual MGXS calculation using built-in class.}, label={lst:mgxs-manual}]{listings/mgxs-manual.py}

\lstinputlisting[language=Python, basicstyle=\ttfamily\scriptsize, caption={Manual MGXS calculation using \texttt{Library} class.}, label={lst:mgxs-library}]{listings/mgxs-library.py}


%%%%%%%%%%%%%%%%%%%%%%%%%%%%%
\subsection{Data Processing}
\label{subsec:data-processing}

\begin{itemize}[noitemsep]
\item refer to PyAPI features that are ``inherited'' by \texttt{openmc.mgxs}:
  \begin{itemize}[noitemsep]
  \item Pandas DataFrames, tally arithmetic/slicing/merging, ...
  \end{itemize}
\end{itemize}

%%%%%%%%%%%%%%%%%%%%%%%%%
\subsection{Data Storage}
\label{subsec:data-storage}

The \texttt{openmc.mgxs} module was developed with general design principles to generate MGXS for any multi-group neutron transport code. Although the module does not explicitly support any multi-group codes, it can export MGXS data to a variety of data storage formats, including Comma-Separated Values (CSV) and HDF5. The exported MGXS files may be easily transformed into the database or input files required by a particular multi-group code.

%%%%%%%%%%%%%%%%%%%%%%%%%%%%%%%%%%%%%%%%%%%%%%%%%%%%%%%%%%%%%%%%%%%%%%%%%%%%%%%
\subsection{Features}
\label{sec:features}
%%%%%%%%%%%%%%%%%%%%%%%%%%%%%%%%%%%%%%%%%%%%%%%%%%%%%%%%%%%%%%%%%%%%%%%%%%%%%%%


%%%%%%%%%%%%%%%%%%%%%%%%%%%%%%%%%%%%%%%%%%%%
\subsubsection{Energy Condensation}
\label{subsec:energy-condense}

The module supports energy condensation in downstream data processing which is useful for exploring approximation bias in various energy group structures. For example, MGXS may be computed in some ``fine'' energy group structure and the tally data subsequently condensed to some coarser group structure \texttt{coarse_groups} for multi-group calculations with the \texttt{MGXS.get_condensed_xs(coarse_groups)} Python class method. Energy condensation may be performed to arbitrarily defined coarse group structures provided the coarse group boundaries coincide with boundaries in the fine group structure.


%%%%%%%%%%%%%%%%%%%%%%%%%%%%%%%%%%%%%%%%%%%%
\subsubsection{Pin-Wise Spatial Homogenization}
\label{subsec:pinwise-homogenize}

The \texttt{openmc.mgxs} module is designed to perform spatial homogenization on heterogeneous tally meshes for fine-mesh transport codes. In OpenMC parlance, MGXS may be computed for material, cell or universe spatial domains. In addition, the module supports MGXS calculations for repeated cell instances using distribcell spatial tally domains\cite{lax2014distribcell}. Spatial homogenization across some subset of distributed cell instances \texttt{cell_instances} can be performed using the \texttt{MGXS.get_subdomain_avg_xs(cell_instances)} Python class method.

%The \texttt{openmc.mgxs} module may also perform spatial homogenization on structured Cartesian tally meshes for coarse mesh multi-group calculations..


%%%%%%%%%%%%%%%%%%%%%%%%%%%%%%%%%%%%%%%%%%%%
\subsubsection{Microscopic MGXS}
\label{subsec:micro-macro}

I hear it is unique that we can calculate micros?


%%%%%%%%%%%%%%%%%%%%%%%%%%%%%%%%%%%%%%%%
\subsubsection{Isotropic in Lab Scattering}
\label{subsec:iso-in-lab}

A unique option for isotropic in lab scattering is implemented in the OpenMC code. The isotropic in lab feature, abbreviated as \emph{iso-in-lab}, may be useful to quantify the ability of multi-group codes to capture anisotropic scattering effects with higher order scattering matrices or transport correction schemes. The iso-in-lab feature is implemented as an optional attribute for each nuclide or element in a simulation. When iso-in-lab scattering is specified for a nuclide or element, the outgoing neutron energy is sampled from the scattering laws prescribed by the continuous energy cross section library, but the outgoing neutron direction of motion is sampled from an isotropic in lab distribution.

%%%%%%%%%%%%%%%%%%%%%%%%%%%%%
\subsection{Tally Estimators}
\label{subsec:tally-est}

\begin{itemize}[noitemsep]
\item Table of the tally estimators acceptable for each MGXS type
\item Mention tally estimators can be toggled? Or discuss in software design?
\item Add footnote mentioning consistent scattering formulation
\end{itemize}

\begin{table}[h!]
  \centering
  \caption{The tally estimators supported by each MGXS type.}
  \small
  \label{tab:mgxs-tally-estimators}
  \vspace{6pt}
  \begin{tabular}{l c c c}
  \toprule
  \textbf{Class} &
  \textbf{Analog} &
  \textbf{Collision} &
  \textbf{Track-length} \\
  \midrule
  \multicolumn{4}{c}{\bf Prompt Neutron Constants} \\
  \midrule
  \texttt{AbsorptionXS} & \cmark & \cmark & \cmark \\
  \texttt{CaptureXS} & \cmark & \cmark & \cmark \\
  \texttt{Chi} & \cmark & \xmark & \xmark \\
  \texttt{FissionXS} & \cmark & \cmark & \cmark \\
  \texttt{InverseVelocity} & \cmark & \cmark & \cmark \\
  \texttt{KappaFissionXS} & \cmark & \cmark & \cmark \\
  \texttt{MultiplicityMatrixXS} & \cmark & \xmark & \xmark \\
  \texttt{NuFissionMatrixXS} & \cmark & \xmark & \xmark \\
  \texttt{ScatterXS} & \cmark & \cmark & \cmark \\
  \texttt{ScatterMatrixXS} & \cmark & \xmark & \xmark \\
  \texttt{ScatterProbabilityMatrixXS} & \cmark & \xmark & \xmark \\
  \texttt{TotalXS} & \cmark & \cmark & \cmark \\
  \texttt{TransportXS} & \cmark & \xmark & \xmark \\
  \midrule
  \multicolumn{4}{c}{\bf Delayed Neutron Constants} \\
  \midrule
  \texttt{Beta} & \cmark & \cmark & \cmark \\
  \texttt{ChiDelayed} & \cmark & \xmark & \xmark \\
  \texttt{DecayRate} & \cmark & \cmark & \cmark \\
  \texttt{DelayedNuFissionXS} & \cmark & \cmark & \cmark \\
  \texttt{DelayedNuFissionMatrixXS} & \cmark & \xmark & \xmark \\
  \bottomrule
\end{tabular}
\end{table}
