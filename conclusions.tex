%%%%%%%%%%%%%%%%%%%%%%%%%%%%%%%%%%%%%%%%%%%%%%%%%%%%%%%%%%%%%%%%%%%%%%%%%%%%%%%
\section{Conclusions}
\label{sec:conclusions}
%%%%%%%%%%%%%%%%%%%%%%%%%%%%%%%%%%%%%%%%%%%%%%%%%%%%%%%%%%%%%%%%%%%%%%%%%%%%%%%

Monte Carlo methods are increasingly used to generate multi-group cross sections for coarse mesh neutron diffusion codes. This paper introduced the \texttt{openmc.mgxs} Python module to generate MGXS with the OpenMC Monte Carlo code for neutron transport applications. This new module utilizes OpenMC's tally system to perform stochastic integration of reaction rates and fluxes for a user-specified energy group discretization and spatial mesh. The different types of MGXS computable to date --- including standard group-wise constants, as well as prompt and delayed constants --- were tabulated here. The \texttt{openmc.mgxs} module leverages OpenMC's Python API, along with its support for tally slicing, merging and arithmetic, to provide a scalable data processing framework complementary to but separate from the transport kernel implemented in the OpenMC executable. A case study used the multi-group OpenMOC transport code with 70-group MGXS generated by OpenMC to model two heterogeneous PWR benchmarks. The study showed that OpenMOC predicted eigenvalues to within 50 pcm and fission rates to within 1\% of reference solutions computed by OpenMC, demonstrating the efficacy of the \texttt{openmc.mgxs} module to enable highly accurate multi-group transport calculations. It is the authors' hope that \texttt{openmc.mgxs} may prove to be a useful platform for future research in the area of MC-based MGXS generation.

%A fine spatial and/or energy discretization may be used by OpenMC to tabulate MGXS, and subsequently condensed in energy and/or homogenized in space with data processing features in Python.
